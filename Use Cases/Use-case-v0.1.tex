\documentclass[12pt,a4paper,twoside]{book}
% Language setup for Greek and English
\usepackage[english,greek]{babel}
\usepackage[utf8]{inputenc}
\usepackage[T1,LGR]{fontenc}
% Font setup
\usepackage{fontspec}
\setmainfont{Linux Libertine O}[Scale=0.9]
\setsansfont{Linux Biolinum}[Scale=0.9]
\setmonofont{DejaVu Sans Mono}[Scale=0.9]
% Other necessary packages
\usepackage{enumitem}
\usepackage{graphicx}
\usepackage{amsmath}
\usepackage{makeidx}
\usepackage{multicol}
\usepackage{multirow}
\usepackage{hanging}
\usepackage{adjustbox}
\usepackage{amssymb}
\usepackage{stackengine}
\usepackage{ifthen}
\usepackage{array}
\usepackage{tcolorbox}
\tcbuselibrary{skins,breakable}
\usepackage{minted}
\usepackage{listings}
\usepackage{parskip}
\usepackage{float}
\usepackage{subcaption}
\usepackage{xcolor}
\usepackage{cancel}
\usepackage{pdfpages}
\usepackage{titlesec}
% Page & text layout
\usepackage{geometry}
\geometry{
   a4paper,
   top=2.5cm,
   bottom=2.5cm,
   left=2.5cm,
   right=2.5cm
}
% Headers & footers
\usepackage{fancyhdr}
\pagestyle{fancyplain}
\fancyhf{} % Clear all header/footer fields

% Define plain style for chapter pages and TOC
\fancypagestyle{plain}{%
    \fancyhf{} % Clear all header/footer fields
    \renewcommand{\headrulewidth}{0pt} % Remove header rule
    \renewcommand{\footrulewidth}{0pt} % Remove footer rule
}

% Define TOC style with specific headers
\fancypagestyle{tocstyle}{%
    \fancyhf{} % Clear all header/footer fields
    \fancyhead[LE]{\thepage}
    \fancyhead[CE]{\leftmark}
    \fancyhead[CO]{\rightmark}
    \fancyhead[RO]{\thepage}
    \renewcommand{\headrulewidth}{0pt}
}

% Redefine \tableofcontents to use tocstyle
\let\oldtableofcontents\tableofcontents
\renewcommand{\tableofcontents}{%
    \clearpage
    % First page of TOC should be empty
    \thispagestyle{empty}
    \pagestyle{tocstyle}
    \oldtableofcontents % chktex 1
    \clearpage
    \pagestyle{fancy}
    % Regular pages header settings
    \fancyhead[LE]{\thepage} % Left header on Even pages: Shows page number
    \fancyhead[CE]{\leftmark} % Center header on Even pages: Shows chapter name (stored in \leftmark)
    \fancyhead[RE]{ΚΕΦ. \thechapter} % Right header on Even pages: Shows "ΚΕΦ." followed by chapter number % chktex 12

    \fancyhead[LO]{\thesection} % Left header on Odd pages: Shows current section number
    \fancyhead[CO]{\rightmark} % Center header on Odd pages: Shows section name (stored in \rightmark)
    \fancyhead[RO]{\thepage} % Right header on Odd pages: Shows page number

    \renewcommand{\headrulewidth}{0.4pt} % Sets thickness of the horizontal line below the header

    % Redefine chapter mark to remove "Chapter" prefix
    \renewcommand{\chaptermark}[1]{\markboth{#1}{}}
    % Redefine section mark to use section title only
    \renewcommand{\sectionmark}[1]{\markright{#1}}
}

% Make table of contents use plain style
\addtocontents{toc}{\protect\thispagestyle{plain}}

% Hyperlinks
\usepackage{hyperref}
\usepackage{bookmark}                                       % Add this line to fix the hyperref warnings
\hypersetup{
    colorlinks=true,
    linkcolor=blue,
    citecolor=blue,
    filecolor=blue,
    urlcolor=cyan,
    unicode,
    pdftitle={1ο Project Λειτουργικών Συστημάτων},
    pdfsubject={Αναφορά Εργασίας},
    pdfborderstyle={/S/U/W 1},                              % Underline links
}
\usepackage{tikz}
\usetikzlibrary{positioning, shapes.geometric, arrows.meta}
\usepackage{pgfplots}
\pgfplotsset{compat=1.18}
\usetikzlibrary{calc,arrows.meta,positioning}

\definecolor{customRed}{HTML}{FF5F5A}                       % using hexadecimal
\definecolor{customYellow}{HTML}{FFBE2E}                    % using hexadecimal
\definecolor{customGreen}{HTML}{2ACA44}                     % using hexadecimal
\definecolor{customPurple}{HTML}{C477DB}                    % using hexadecimal
\definecolor{customBlue}{HTML}{052538}                      % using hexadecimal
\definecolor{customLightBlue}{HTML}{60ADEC}                 % using hexadecimal
\definecolor{customGray}{HTML}{A0A0A0}                      % using hexadecimal
\definecolor{customBlack}{HTML}{000000}                     % using hexadecimal

% Define a lighter gray for background and darker colors for text
\definecolor{customBgGray}{HTML}{E0E0E0}         % Light gray background
\definecolor{customCodeRed}{HTML}{C92A2A}        % Darker red
\definecolor{customCodeYellow}{HTML}{E67700}     % Darker yellow/orange
\definecolor{customCodeGreen}{HTML}{087F23}      % Darker green
\definecolor{customCodePurple}{HTML}{7B1FA2}     % Darker purple
\definecolor{customCodeBlue}{HTML}{1976D2}       % Darker blue
\definecolor{customCodeBlack}{HTML}{1A1A1A}      % Lighter black

\lstdefinestyle{myPythonStyle}{
    language=Python,                                            % Set the language to Python
    backgroundcolor=\color{customBgGray},                       % Background color
    basicstyle=\color{customCodeBlack}\ttfamily\footnotesize,   % Basic font style, size, and color
    keywordstyle=\color{customCodePurple}\bfseries,             % Style for general keywords
    keywordstyle=[2]\color{customCodeYellow}\bfseries,          % Style for specific keywords
    commentstyle=\color{customCodeGreen}\itshape,               % Style for comments
    stringstyle=\color{customCodeGreen},                        % Style for strings
    identifierstyle=\color{customCodeBlue},                     % Style for identifiers
    morekeywords={self, None, True, False, format, abs,
                as, pass, return, if, elif, else, for,
                range, while, try, except, with, lambda,
                yield, global, nonlocal, assert, del,
                raise, in, is, and, or, not},                   % Additional keywords
    morekeywords=[2]{os, graphviz, Digraph, collections,
                    deque, math, tabulate},                     % Define 'import' and 'from' as additional keywords
    % procnamekeys={def,class},                                 % Highlight function and class names
    showstringspaces=false,                                     % Don't show spaces in strings
%
    breaklines=true,                                            % Automatically break long lines
    breakatwhitespace=false,                                    % Break lines at any character
    % prebreak=\textbackslash,                                  % Character for breaking lines
    postbreak={\space},                                         % Character for breaking lines
    breakautoindent=false,                                      % Indentation after line break
    breakindent=0pt,                                            % Indentation before line break
    resetmargins=true,                                          % Reset margins after line break
    keepspaces=true,                                            % Keep spaces in the code
    showspaces=false,                                           % Don't show spaces
    columns=flexible,                                           % Column format
%
    numbers=left,                                               % Line numbers on the left
    numberstyle=\small\color{customBgGray},                     % Style for line numbers
    % numwidth=4em,                                             % Width allocated for line numbers
    stepnumber=1,                                               % Step between line numbers
    numbersep=8pt,                                              % Distance of line numbers from code
    xleftmargin=1.5em,                                          % Match the numwidth
    tabsize=4,                                                  % Size of a tab
    captionpos=t,                                               % Position of the caption (t for top)
    firstnumber=auto,                                           % Continue line numbering from previous listing
    aboveskip=0em,                                              % Remove external spacing
    belowskip=0em,                                              % Remove external spacing
    frame=none,                                                 % Add top and bottom rules only
    framerule=0pt,                                              % Make the rules invisible
    rulecolor=\color{customBgGray},                             % Color of the frame (if enabled)
    framesep=2pt,                                               % Add internal padding
    % framexleftmargin=1em,                                     % Add internal left margin
    % framexrightmargin=1em,                                    % Add internal right margin
    escapeinside={\(*@}{@*\)},                                  % For escaping characters
    morecomment=[l]{\#},                                        % Define comment style
    morestring=[b]',                                            % Define string style with single quotes
    morestring=[b]",                                            % Define string style with double quotes % chktex 18
    literate={
        {``}{{\textquotedbleft}}2
        {''}{{\textquotedbright}}2
        {`}{{\textquoteleft}}1
        {'}{{\textquoteright}}1
        {_}{{\_}}1
    },
}

\lstdefinestyle{myCStyle}{
    language=C,
    backgroundcolor=\color{customBgGray},                       % Background color
    basicstyle=\color{customCodeBlack}\ttfamily\footnotesize,   % Basic font style
    keywordstyle=\color{customCodePurple}\bfseries,            % Style for keywords
    keywordstyle=[2]\color{customCodeYellow}\bfseries,         % Style for additional keywords
    commentstyle=\color{customCodeGreen}\itshape,              % Style for comments
    stringstyle=\color{customCodeGreen},                       % Style for strings
    identifierstyle=\color{customCodeBlue},                    % Style for identifiers
    morekeywords={void, int, char, float, double, long, short, signed, unsigned,
                  const, static, extern, volatile, register, auto, struct, union,
                  typedef, enum, sizeof, break, continue, goto, return, if, else,
                  switch, case, default, for, while, do, pid_t, sem_t},
    morekeywords=[2]{stdio.h, stdlib.h, string.h, math.h, time.h, ctype.h,
                     unistd.h, semaphore.h, sys/mman.h, sys/wait.h, fcntl.h,
                     printf, fprintf, perror, exit, malloc, free, fork, waitpid,
                     mmap, munmap, sem_init, sem_wait, sem_post, sem_destroy,
                     sleep, rand, srand, atoi, time, MAP_SHARED, MAP_ANONYMOUS,
                     PROT_READ, PROT_WRITE, MAP_FAILED, EXIT_SUCCESS, EXIT_FAILURE},
    showstringspaces=false,
    frame=none,
    rulecolor=\color{customBgGray},
    breaklines=true,                % Break long lines
    breakatwhitespace=false,        % Break at any character
    postbreak={\space},             % Character after break
    breakautoindent=false,          % No indentation after break
    breakindent=0pt,                % No indent before break
    resetmargins=true,              % Reset margins after break
    keepspaces=true,                % Preserve spaces
    showspaces=false,               % Don't show spaces
    columns=flexible,               % Column format
    numbers=left,
    numberstyle=\small\color{customBgGray},
    stepnumber=1,
    numbersep=8pt,
    xleftmargin=1.5em,
    tabsize=4,
    captionpos=t,
    firstnumber=auto,
    aboveskip=0em,
    belowskip=0em,
    framesep=2pt,
    escapeinside={\(*@}{@*\)},
    morecomment=[l]{//},
    morecomment=[s]{/*}{*/},
    morestring=[b]", % chktex 18
    morestring=[b]',
    literate={
        {``}{{\textquotedbleft}}2
        {''}{{\textquotedbright}}2
        {`}{{\textquoteleft}}1
        {'}{{\textquoteright}}1
        {_}{{\_}}1
    },
}

\lstdefinestyle{myBashStyle}{
    language=bash,
    backgroundcolor=\color{customBgGray},                       % Background color
    basicstyle=\color{customCodeBlack}\ttfamily\footnotesize,   % Basic font style
    keywordstyle=\color{customCodePurple}\bfseries,            % Style for keywords
    keywordstyle=[2]\color{customCodeYellow}\bfseries,         % Style for additional keywords
    commentstyle=\color{customCodeGreen}\itshape,              % Style for comments
    stringstyle=\color{customCodeGreen},                       % Style for strings
    identifierstyle=\color{customCodeBlue},                    % Style for identifiers
    alsoletter={0123456789}                                    % Define numbers as letters
    morekeywords={if, then, else, elif, fi, for, while, do, done, in, case, esac, 
                  function, select, until, break, continue, return, exit, shift,
                  declare, local, readonly, export, set, unset},
    morekeywords=[2]{echo, read, cat, awk, grep, sed, cut, tr, sort, uniq, wc,
                     mkdir, rm, cp, mv, ls, cd, pwd, touch, chmod, chown, find,
                     test, source, alias, eval, exec, getopts, printf, wait,
                     STDIN, STDOUT, STDERR, IFS, PATH, BEGIN, END, NR, NF, 
                     tolower, concat, print},
    showstringspaces=false,
    frame=none,
    rulecolor=\color{customBgGray},
    breaklines=true,                % Break long lines
    breakatwhitespace=false,        % Break at any character
    postbreak={\space},             % Character after break
    breakautoindent=false,          % No indentation after break
    breakindent=0pt,                % No indent before break
    resetmargins=true,              % Reset margins after break
    keepspaces=true,                % Preserve spaces
    showspaces=false,               % Don't show spaces
    columns=flexible,               % Column format
    numbers=left,
    numberstyle=\small\color{customBgGray},
    stepnumber=1,
    numbersep=8pt,
    xleftmargin=1.5em,
    tabsize=4,
    captionpos=t,
    firstnumber=auto,
    aboveskip=0em,
    belowskip=0em,
    framesep=2pt,
    escapeinside={\(*@}{@*\)},
    morecomment=[l]{\#},
    morestring=[b]", % chktex 18
    morestring=[b]',
    literate={
        {``}{{\textquotedbleft}}2
        {''}{{\textquotedbright}}2
        {`}{{\textquoteleft}}1
        {'}{{\textquoteright}}1
        {_}{{\_}}1
    },
}

\tcbset{
    myCustomStyle/.style={
        enhanced,                       % Enable enhanced features
        breakable,                      % Allow the box to break across pages
        colback=customBlue,             % Background color
        colframe=customBgGray,              % Frame color
        listing only,                   % Use listings
%
        fonttitle=\bfseries,            % Title font style
        % frame=single,                 % Frame type
        % interior titled,              % Use regular titled interior
        interior style={
            top color=black!75,         % Title area background
            bottom color=black!75,      % Title area background
        },
        colbacktitle=black!75,          % Title background color
        coltitle=white,                 % Title color
        rounded corners,                % Rounded corners
        boxrule=1mm,                    % Frame thickness
        drop shadow southeast,          % Shadow effect
        lefttitle=0pt,                  % Remove left margin of title
        % left=1em,                     % Increase left margin for line numbers
        title={\hspace*{-1em}\textcolor{customRed}{● }\textcolor{customYellow}{● }\textcolor{customGreen}{●}\quad#1}, % Circles + Title
        attach title to upper={\vspace{2.3mm}}, % Adjust vertical spacing
%
        beforeafter skip=8pt,           % Space before and after the box
        breaklines=true,                % Allow the box to break across pages
        breakatwhitespace=false,        % Break lines at any character if necessary
        % sharp corners                 % Sharp corners for the title area
    }
}

% Custom commands for language switching
\newcommand{\en}[1]{\foreignlanguage{english}{#1}}
\newcommand{\gr}[1]{\foreignlanguage{greek}{#1}}
% Custom command for coloring
\newcommand{\blue}[1]{\textcolor{blue}{#1}}
% Index
\makeindex
% Set headheight to avoid fancyhdr warnings
\setlength{\headheight}{14.5pt}

% Adjust chapter spacing
\titleformat{\chapter}[display]
{\normalfont\huge\bfseries}{\chaptertitlename\ \thechapter}{15pt}{\Huge}
\titlespacing*{\chapter}{0pt}{-10pt}{20pt}

% Adjust section spacing
\titlespacing*{\section}{0pt}{10pt}{10pt}

% Adjust vertical space between paragraphs
\setlength{\parskip}{0.5em}

% Make figure placement less strict
\setcounter{topnumber}{2}
\setcounter{bottomnumber}{2}
\setcounter{totalnumber}{4}
\renewcommand{\topfraction}{0.85}
\renewcommand{\bottomfraction}{0.85}
\renewcommand{\textfraction}{0.15}
\renewcommand{\floatpagefraction}{0.7}

\begin{document}
% Start Arabic numbering from the beginning but hide numbers
\pagenumbering{arabic}
% Add this command to prevent blank pages
\let\cleardoublepage\clearpage

% Set main language to Greek
\selectlanguage{greek}

% Title page with hidden number
\begin{titlepage}
  \vspace*{8cm}
  \begin{center}
    {\LARGE \textbf{\textit{Pet-à-Vet}}}\\ % chktex 8
    \vspace*{0.5cm}
    {\large Use Cases}\\
    {\large Τεχνολογία Λογισμικού}
    \vspace*{5cm}
    \\
    \vspace*{1cm}
    \begin{tabular}{l l l}
      % \large Βασίλειος Μπίτζας            & \large 1083796 & \large up1083796@ac.upatras.gr \\
      % \large Αριάδνη Σβωλοπούλου          & \large 1104806 & \large up1104806@ac.upatras.gr \\
    \end{tabular}

    \vspace*{1cm}

    {\normalsize Τμήμα Μηχανικών Η/Υ \& Πληροφορικής}
    \\
    {\normalsize Πανεπιστήμιο Πατρών}
    \\
  \end{center}
  \thispagestyle{empty} % Suppress page number on the title page
\end{titlepage}

% % Add this command to prevent blank pages
% \let\cleardoublepage\clearpage
% \frontmatter
\tableofcontents
% \mainmatter % chktex 1

\printindex

\chapter{Use Cases}

\section{Customer management}

\subsection{Σύντομη Περιγραφή}
Η περίπτωση χρήσης "Customer Management" επιτρέπει στους διαχειριστές της κτηνιατρικής κλινικής και στο προσωπικό υποδοχής να διαχειρίζονται πλήρως τα προφίλ των πελατών (ιδιοκτητών κατοικίδιων). Αυτό περιλαμβάνει την καταχώρηση νέων πελατών, την ενημέρωση των στοιχείων τους, την προβολή του ιστορικού επισκέψεων, την αντιστοίχιση με τα κατοικίδιά τους, και την παρακολούθηση οικονομικών συναλλαγών.

\subsection{Χειριστές}
\begin{itemize}
  \item Διαχειριστής συστήματος
  \item Κτηνίατρος
  \item Γραμματέας
\end{itemize}

\subsection{Γεγονός Έναρξης}
Η περίπτωση χρήσης ξεκινά όταν:
\begin{itemize}
  \item Ένας διαχειριστής, κτηνίατρος ή υπάλληλος υποδοχής επιλέγει την επιλογή "Customer Management" από το κεντρικό μενού
        %   \item Ένας νέος πελάτης εγγράφεται στο σύστημα μέσω της διαδικτυακής πλατφόρμας
        %   \item Το προσωπικό χρειάζεται να αναζητήσει πληροφορίες για έναν πελάτη
        %   \item Ο ιδιοκτήτης κατοικίδιου συνδέεται στο λογαριασμό του για να επεξεργαστεί τα στοιχεία του
\end{itemize}

\subsection{Ροή γεγονότων}

\subsubsection{Βασική Ροή}
\begin{enumerate}
  \item Ο χρήστης μπαίνει στην ενότητα "Customer Management".
  \item Το σύστημα εμφανίζει μια λίστα με τους υπάρχοντες πελάτες.
  \item Ο χρήστης επιλέγει τη προσθήκη πελάτη.
  \item Το σύστημα εμφανίζει φόρμα με τα απαραίτητα πεδία %(ονοματεπώνυμο, στοιχεία επικοινωνίας, διεύθυνση κτλ.).
  \item Ο χρήστης συμπληρώνει τα στοιχεία και υποβάλλει τη φόρμα.
  \item Το σύστημα επικυρώνει τα δεδομένα, δημιουργεί και αποθηκεύει το νέο προφίλ πελάτη.
  \item Το σύστημα αποστέλλει αυτόματα email/SMS καλωσορίσματος με οδηγίες ενεργοποίησης λογαριασμού.
  \item Το σύστημα οδηγεί τον χρήστη στη βήμα 2 της βασικής ροής.
\end{enumerate}

\subsubsection{Εναλλακτικές Ροές}
\begin{enumerate}
  \item[1 ] "Search"(/προβολή) υπάρχοντος πελάτη:
        \begin{enumerate}
          \item[1.1 ] Ο χρήστης επιλέγει την επιλογή αναζήτησης
          \item[1.2 ] Ο χρήστης εισάγει κριτήρια αναζήτησης %(όνομα, email, τηλέφωνο, κ.λπ.)
          \item[1.3 ] Το σύστημα εμφανίζει τα αποτελέσματα που ταιριάζουν
          \item[1.4 ] Ο χρήστης επιλέγει την προβολή του επιθυμητού πελάτη από τα αποτελέσματα και προβάλει τα στοχεία του
          \item[1.5 ] Ο χρήστης κλείνει το παράθυρο αναζήτησης και επιστρέφει στο βήμα 2 της βασικής ροής
        \end{enumerate}
  \item[2 ] Επεξεργασία προφίλ πελάτη:
        \begin{enumerate}
          \item [2.1 ] Ο χρήστης επιλέγει την επιλογή αναζήτησης
          \item [2.2 ] Ο χρήστης εισάγει κριτήρια αναζήτησης %(όνομα, email, τηλέφωνο, κ.λπ.)
          \item [2.3 ] Το σύστημα εμφανίζει τα αποτελέσματα που ταιριάζουν
          \item [2.4 ] Ο χρήστης επιλέγει την επεξεργασία του επιθυμητού πελάτη από τα αποτελέσματα και επεξεργάζεται τα στοιχεία του
          \item [2.5 ] Ο χρήστης αποθηκεύει τις αλλαγές και επιστρέφει στο βήμα 2 της βασικής ροής
        \end{enumerate}
  \item[3 ] "Sort" πελατών:
        \begin{enumerate}
          \item [3.1 ] Ο χρήστης επιλέγει την επιλογή ταξινόμησης
          \item [3.2 ] Ο χρήστης επιλέγει το κριτήριο ταξινόμησης %(όνομα, ημερομηνία εγγραφής, κ.λπ.)
          \item [3.3 ] Το σύστημα ταξινομεί τους πελάτες σύμφωνα με το επιλεγμένο κριτήριο και τους εμφανίζει στην οθόνη
        \end{enumerate}
  \item[4 ] "Filter" πελατών:
        \begin{enumerate}
          \item [4.1 ] Ο χρήστης επιλέγει την επιλογή φιλτραρίσματος
          \item [4.2 ] Ο χρήστης επιλέγει το κριτήριο φιλτραρίσματος %(κατάσταση λογαριασμού, ημερομηνία εγγραφής, κατοικίδιο, κ.λπ.)
          \item [4.3 ] Ο χρήστης εισάγει τις τιμές του κριτηρίου φιλτραρίσματος
          \item [4.4 ] Το σύστημα εφαρμόζει το φίλτρο στη λίστα των πελατών και τους εμφανίζει στην οθόνη
        \end{enumerate}
  \item[5 ] Λανθασμένη εισαγωγή δεδομένων:
        \begin{enumerate}
          \item [5.1 ] Ο χρήστης επιλέγει την επιλογή προσθήκης
          \item [5.2 ] Ο χρήστης εισάγει τα στοιχεία του νέου πελάτη σε λανθασμένη μορφή
          \item [5.3 ] Το σύστημα εμφανίζει μήνυμα λάθους και οδηγεί τον χρήστη στο βήμα 4 της βασικής ροής
        \end{enumerate}
  \item[6 ] Ακύρωση προσθήκης:
        \begin{enumerate}
          \item [6.1 ] Ο χρήστης επιλέγει την επιλογή προσθήκης
          \item [6.2 ] Ο χρήστης εισάγει τα στοιχεία του νέου πελάτη
          \item [6.3 ] Ο χρήστης επιλέγει την επιλογή ακύρωσης
          \item [6.4 ] Το σύστημα εμφανίζει μήνυμα επιβεβαίωσης και οδηγεί τον χρήστη στο βήμα 2 της βασικής ροής
        \end{enumerate}
  \item[7 ] Ακύρωση επεξεργασίας:
        \begin{enumerate}
          \item [7.1 ] Ο χρήστης επιλέγει την επιλογή αναζήτησης
          \item [7.2 ] Ο χρήστης εισάγει κριτήρια αναζήτησης %(όνομα, email, τηλέφωνο, κ.λπ.)
          \item [7.3 ] Το σύστημα εμφανίζει τα αποτελέσματα που ταιριάζουν
          \item [7.4 ] Ο χρήστης επιλέγει την επεξεργασία του επιθυμητού πελάτη από τα αποτελέσματα
          \item [7.5 ] Ο χρήστης επιλέγει την επιλογή ακύρωσης
          \item [7.6 ] Το σύστημα εμφανίζει μήνυμα επιβεβαίωσης και οδηγεί τον χρήστη στο βήμα 2 της βασικής ροής
        \end{enumerate}
  \item[8 ] Λανθασμένα κριτήρια αναζήτησης:
        \begin{enumerate}
          \item [8.1 ] Ο χρήστης επιλέγει την επιλογή αναζήτησης
          \item [8.2 ] Ο χρήστης εισάγει λανθασμένα κριτήρια αναζήτησης
          \item [8.3 ] Το σύστημα εμφανίζει μήνυμα λάθους και οδηγεί τον χρήστη στο βήμα 2 της εναλλακτικής ροής
        \end{enumerate}
  \item[9 ] Πάτημα πελάτη:
        \begin{enumerate}
          \item [9.1 ] Ο χρήστης επιλέγει έναν πελάτη από αυτούς που εμφανίζονται αρχικά στην οθόνη
          \item [9.2 ] Το σύστημα εμφανίζει τα στοιχεία του επιλεγμένου πελάτη
          \item [9.3 ] Ο χρήστης πατάει κλείσιμο και το σύστημα τον γυρνάει στο βήμα 2 της βασικής ροής
        \end{enumerate}
        %   \item[10 ] Εξαγωγή στατιστικών:
        %     \begin{enumerate}
        %         \item [10.1 ] Ο χρήστης επιλέγει την επιλογή εξαγωγής
        %         \item [10.2 ] Ο χρήστης επιλέγει το είδος των στατιστικών που επιθυμεί (πλήθος πελατών, κατανομή κατοικιδίων, κ.λπ.)
        %         \item [10.3 ] Το σύστημα εμφανίζει τα στατιστικά στοιχεία στην οθόνη και παρέχει τη δυνατότητα εξαγωγής σε αρχείο
        %     \end{enumerate}
\end{enumerate}

\subsection{Ειδικές απαιτήσεις}
\begin{itemize}
  \item Το σύστημα πρέπει να συμμορφώνεται με τους κανονισμούς GDPR για την προστασία προσωπικών δεδομένων
  \item Τα δεδομένα των πελατών θα πρέπει να κρυπτογραφούνται κατά την αποθήκευση
  \item Η πρόσβαση στα οικονομικά στοιχεία περιορίζεται σε χρήστες με ειδική εξουσιοδότηση
  \item Το σύστημα πρέπει να διατηρεί αρχείο καταγραφής όλων των τροποποιήσεων (audit trail)
  \item Υποστήριξη μαζικής εισαγωγής πελατών μέσω αρχείου CSV
\end{itemize}

\subsubsection{Μη λειτουργικές απαιτήσεις}
\begin{itemize}
  \item Ο χρόνος απόκρισης για αναζήτηση πελατών πρέπει να είναι λιγότερος από 2 δευτερόλεπτα ακόμα και με βάση δεδομένων 10.000+ πελατών
  \item Το σύστημα πρέπει να παραμένει λειτουργικό ακόμα και με περιορισμένη συνδεσιμότητα στο διαδίκτυο, διατηρώντας τοπικό αντίγραφο των πιο πρόσφατων συναλλαγών
  \item Το περιβάλλον χρήσης πρέπει να είναι προσβάσιμο από άτομα με αναπηρίες (WCAG 2.1 επίπεδο AA)
  \item Το σύστημα πρέπει να υποστηρίζει πολλαπλές γλώσσες (Ελληνικά και Αγγλικά αρχικά)
\end{itemize}

\subsubsection{Περιβάλλον}
\begin{itemize}
  \item Διαδικτυακή πλατφόρμα προσβάσιμη μέσω φυλλομετρητών (Chrome, Firefox, Safari, Edge)
  \item Εφαρμογή για κινητές συσκευές (iOS και Android)
        %   \item Διασύνδεση με συστήματα τιμολόγησης και λογιστικά προγράμματα
        %   \item Εκτυπωτές για αποδείξεις και έγγραφα
\end{itemize}

\subsection{Κατάσταση εισόδου}
\begin{itemize}
  \item Ο χρήστης έχει συνδεθεί στο σύστημα με τα κατάλληλα διαπιστευτήρια
  \item Η βάση δεδομένων του συστήματος είναι προσβάσιμη και λειτουργική
\end{itemize}

\subsection{Κάτασταση εξόδου}
\begin{itemize}
  \item Τα στοιχεία του πελάτη έχουν καταχωρηθεί ή ενημερωθεί επιτυχώς στη βάση δεδομένων
  \item Όλες οι ενέργειες έχουν καταγραφεί στο αρχείο ιστορικού του συστήματος
  \item Έχουν σταλεί οι κατάλληλες ειδοποιήσεις στους εμπλεκόμενους (π.χ. email επιβεβαίωσης στον πελάτη)
  \item Ο χρήστης επιστρέφει στην κεντρική οθόνη διαχείρισης πελατών % ή προχωρά σε σχετική λειτουργία (π.χ. προγραμματισμός ραντεβού)
\end{itemize}

\subsection{Σημεία επέκτασης}
% \begin{itemize}
%     \item Διασύνδεση με σύστημα αυτόματης αποστολής υπενθυμίσεων για εμβολιασμούς και προληπτικούς ελέγχους
%     \item Ενσωμάτωση συστήματος πιστότητας (loyalty) με πόντους και εκπτώσεις
%     \item Ανάπτυξη εξατομικευμένων προτάσεων υπηρεσιών και προϊόντων βάσει του προφίλ του πελάτη
%     \item Δυνατότητα ψηφιακής υπογραφής εγγράφων συγκατάθεσης για θεραπείες
%     \item Ενσωμάτωση τεχνολογίας αναγνώρισης προσώπου για γρήγορη ταυτοποίηση πελατών στην υποδοχή
% \end{itemize}

\section{Warehouse management}

\subsection{Σύντομη Περιγραφή}
Η περίπτωση χρήσης "Warehouse Management" επιτρέπει στους διαχειριστές της κτηνιατρικής κλινικής και στο εξουσιοδοτημένο προσωπικό να διαχειρίζονται πλήρως το απόθεμα της κλινικής. Αυτό περιλαμβάνει την παρακολούθηση αποθεμάτων φαρμάκων, ιατρικών αναλωσίμων, τροφών και προϊόντων περιποίησης κατοικιδίων, την καταχώρηση νέων προϊόντων, την ενημέρωση ποσοτήτων, την εκτέλεση απογραφών, τη δημιουργία παραγγελιών σε προμηθευτές και την παρακολούθηση ημερομηνιών λήξης.

\subsection{Χειριστές}
\begin{itemize}
  \item Διαχειριστής συστήματος
  \item Κτηνίατρος
  \item Γραμματέας
\end{itemize}

\subsection{Γεγονός Έναρξης}
Η περίπτωση χρήσης ξεκινά όταν:
\begin{itemize}
  \item Ένας διαχειριστής, κτηνίατρος ή υπάλληλος αποθήκης επιλέγει την επιλογή "Warehouse Management" από το κεντρικό μενού
  \item Το σύστημα ανιχνεύει ότι κάποιο προϊόν έχει φτάσει στο ελάχιστο όριο αποθέματος
  \item Πραγματοποιείται παραλαβή νέων προϊόντων από προμηθευτή
  \item Υπάρχει ανάγκη για εκτέλεση περιοδικής απογραφής
\end{itemize}

\subsection{Ροή γεγονότων}

\subsubsection{Βασική Ροή}
\begin{enumerate}
  \item Ο χρήστης μπαίνει στην ενότητα "Warehouse Management" στην καρτέλα "Products".
  \item Το σύστημα εμφανίζει μια λίστα με τα υπάρχοντα προϊόντα και τις ποσότητές τους.
  \item Ο χρήστης επιλέγει την προσθήκη νέου προϊόντος.
  \item Το σύστημα εμφανίζει φόρμα με τα απαραίτητα πεδία %(κωδικός, όνομα, κατηγορία, προμηθευτής, τιμή αγοράς, τιμή πώλησης, ελάχιστο απόθεμα, ημερομηνία λήξης κτλ.).
  \item Ο χρήστης συμπληρώνει τα στοιχεία και υποβάλλει τη φόρμα.
  \item Το σύστημα επικυρώνει τα δεδομένα, δημιουργεί και αποθηκεύει το νέο προϊόν στη βάση δεδομένων.
  \item Το σύστημα ενημερώνει το διαθέσιμο απόθεμα.
  \item Το σύστημα οδηγεί τον χρήστη στο βήμα 2 της βασικής ροής.
\end{enumerate}

\subsubsection{Εναλλακτικές Ροές}
\begin{enumerate}
  \item[1 ] "Search"(/προβολή) υπάρχοντος προϊόντος:
        \begin{enumerate}
          \item[1.1 ] Ο χρήστης επιλέγει την επιλογή αναζήτησης
          \item[1.2 ] Ο χρήστης εισάγει κριτήρια αναζήτησης %(κωδικός, όνομα, κατηγορία, προμηθευτής κ.λπ.)
          \item[1.3 ] Το σύστημα εμφανίζει τα αποτελέσματα που ταιριάζουν
          \item[1.4 ] Ο χρήστης επιλέγει την προβολή του επιθυμητού προϊόντος από τα αποτελέσματα
          \item[1.5 ] Ο χρήστης κλείνει το παράθυρο προβολής και επιστρέφει στο βήμα 2 της βασικής ροής
        \end{enumerate}
  \item[2 ] Ενημέρωση αποθέματος προϊόντος:
        \begin{enumerate}
          \item [2.1 ] Ο χρήστης επιλέγει την επιλογή αναζήτησης
          \item [2.2 ] Ο χρήστης εισάγει κριτήρια αναζήτησης %(κωδικός, όνομα, κατηγορία, προμηθευτής κ.λπ.)
          \item [2.3 ] Το σύστημα εμφανίζει τα αποτελέσματα που ταιριάζουν
          \item [2.4 ] Ο χρήστης επιλέγει την επεξεργασία του επιθυμητού προϊόντος
          \item [2.5 ] Ο χρήστης εισάγει τη νέα ποσότητα και αιτιολογία μεταβολής %(παραλαβή, πώληση, φθορά, λήξη)
          \item [2.6 ] Το σύστημα ενημερώνει το απόθεμα και καταγράφει τη συναλλαγή
          \item [2.7 ] Ο χρήστης επιστρέφει στο βήμα 2 της βασικής ροής
        \end{enumerate}
  \item[3 ] Εκτέλεση απογραφής:
        \begin{enumerate}
          \item [3.1 ] Ο χρήστης επιλέγει την επιλογή "Inventory Check"
          \item [3.2 ] Το σύστημα εμφανίζει φόρμα απογραφής με όλα τα προϊόντα και τις θεωρητικές ποσότητες
          \item [3.3 ] Ο χρήστης καταχωρεί τις πραγματικές ποσότητες που καταμέτρησε
          \item [3.4 ] Το σύστημα υπολογίζει και επισημαίνει τις αποκλίσεις
          \item [3.5 ] Ο χρήστης επιβεβαιώνει την απογραφή
          \item [3.6 ] Το σύστημα ενημερώνει τα αποθέματα και δημιουργεί αναφορά απογραφής
          \item [3.7 ] Ο χρήστης επιστρέφει στο βήμα 2 της βασικής ροής
        \end{enumerate}
  \item[4 ] Δημιουργία παραγγελίας:
        \begin{enumerate}
          \item [4.1 ] Ο χρήστης επιλέγει την καρτέλα "Orders"
          \item [4.2 ] Το σύστημα εμφανίζει τα προϊόντα που βρίσκονται κάτω από το ελάχιστο όριο αποθέματος
          \item [4.3 ] Ο χρήστης επιλέγει προϊόντα και καθορίζει ποσότητες παραγγελίας
          \item [4.4 ] Ο χρήστης επιλέγει προμηθευτή για κάθε προϊόν
          \item [4.5 ] Το σύστημα δημιουργεί παραγγελία και την αποθηκεύει
          \item [4.6 ] Το σύστημα παρέχει επιλογές για άμεση αποστολή της παραγγελίας %μέσω email ή εκτύπωση
          \item [4.7 ] Ο χρήστης επιστρέφει στο βήμα 2 της βασικής ροής
        \end{enumerate}
  \item[5 ] Εκκρεμείς παραγγελίες:
        \begin{enumerate}
          \item [5.1 ] Ο χρήστης επιλέγει την καρτέλα "Orders"
          \item [5.2 ] Ο χρήστης επιλέγει το "Pending Orders"
          \item [5.3 ] Το σύστημα εμφανίζει λίστα εκκρεμών παραγγελιών
          \item [5.4 ] Ο χρήστης επιλέγει την παραγγελία που παραλαμβάνει
          \item [5.5 ] Ο χρήστης καταχωρεί τις παραληφθείσες ποσότητες και τις ημερομηνίες λήξης
          \item [5.6 ] Το σύστημα ενημερώνει το απόθεμα και την κατάσταση της παραγγελίας
          \item [5.7 ] Ο χρήστης επιστρέφει στο βήμα 2 της βασικής ροής
        \end{enumerate}
  \item[6 ] Λανθασμένη εισαγωγή δεδομένων:
        \begin{enumerate}
          \item [6.1 ] Ο χρήστης εισάγει δεδομένα σε λανθασμένη μορφή %(π.χ. αρνητικό απόθεμα, μη έγκυρη ημερομηνία)
          \item [6.2 ] Το σύστημα εμφανίζει μήνυμα λάθους με σχετική επεξήγηση
          \item [6.3 ] Ο χρήστης διορθώνει τα δεδομένα και συνεχίζει τη διαδικασία
        \end{enumerate}
  \item[7 ] Έλεγχος ληγμένων προϊόντων:
        \begin{enumerate}
          \item [7.1 ] Ο χρήστης επιλέγει την επιλογή "Expiration Check"
          \item [7.2 ] Το σύστημα εμφανίζει προϊόντα που πλησιάζουν ή έχουν ξεπεράσει την ημερομηνία λήξης
          \item [7.3 ] Ο χρήστης επιλέγει τα προϊόντα προς απόσυρση
          \item [7.4 ] Το σύστημα καταγράφει την αφαίρεση των προϊόντων από το απόθεμα
          \item [7.5 ] Ο χρήστης επιστρέφει στο βήμα 2 της βασικής ροής
        \end{enumerate}
  \item[8 ] Εξαγωγή αναφορών:
        \begin{enumerate}
          \item [8.1 ] Ο χρήστης επιλέγει την επιλογή "Reports"
          \item [8.2 ] Το σύστημα εμφανίζει τους διαθέσιμους τύπους αναφορών %(τρέχον απόθεμα, ιστορικό κινήσεων, αξία αποθέματος, κ.λπ.)
          \item [8.3 ] Ο χρήστης επιλέγει τον τύπο αναφοράς και το χρονικό διάστημα
          \item [8.4 ] Το σύστημα δημιουργεί και εμφανίζει την αναφορά
          \item [8.5 ] Ο χρήστης έχει τη δυνατότητα εκτύπωσης ή εξαγωγής της αναφοράς
          \item [8.6 ] Ο χρήστης επιστρέφει στο βήμα 2 της βασικής ροής
        \end{enumerate}
  \item[9 ] Ακύρωση παραγγελίας:
        \begin{enumerate}
          \item [9.1 ] Ο χρήστης επιλέγει την καρτέλα "Orders"
          \item [9.2 ] Ο χρήστης επιλέγει την παραγγελία που θέλει να ακυρώσει
          \item [9.3 ] Το σύστημα ζητά επιβεβαίωση για την ακύρωση
          \item [9.4 ] Ο χρήστης επιβεβαιώνει την ακύρωση
          \item [9.5 ] Το σύστημα ενημερώνει την κατάσταση της παραγγελίας και επιστρέφει στο βήμα 2 της βασικής ροής
        \end{enumerate}
  \item[10 ] Ακύρωση προσθήκης προϊόντος:
        \begin{enumerate}
          \item [10.1 ] Ο χρήστης επιλέγει την προσθήκη νέου προϊόντος
          \item [10.2 ] Το σύστημα εμφανίζει φόρμα με τα απαραίτητα πεδία
          \item [10.3 ] Ο χρήστης συμπληρώνει τα στοιχεία και επιλέγει την ακύρωση
          \item [10.4 ] Το σύστημα γυρίζει τον χρήστη στο βήμα 2 της βασικής ροής
        \end{enumerate}
  \item[11 ] Ακύρωση ενημέρωσης προϊόντος:
        \begin{enumerate}
          \item [11.1 ] Ο χρήστης επιλέγει την επιλογή αναζήτησης
          \item [11.2 ] Ο χρήστης εισάγει κριτήρια αναζήτησης %(κωδικός, όνομα, κατηγορία, προμηθευτής κ.λπ.)
          \item [11.3 ] Το σύστημα εμφανίζει τα αποτελέσματα που ταιριάζουν
          \item [11.4 ] Ο χρήστης επιλέγει την επεξεργασία του επιθυμητού προϊόντος
          \item [11.5 ] Ο χρήστης εισάγει τη νέα ποσότητα και αιτιολογία μεταβολής %(παραλαβή, πώληση, φθορά, λήξη)
          \item [11.6 ] Ο χρήστης επιλέγει την ακύρωση της ενημέρωσης
          \item [11.7 ] Το σύστημα γυρίζει τον χρήστη στο βήμα 2 της βασικής ροής
        \end{enumerate}
  \item[12 ] Ακύρωση απογραφής:
        \begin{enumerate}
          \item [12.1 ] Ο χρήστης επιλέγει την επιλογή "Inventory Check"
          \item [12.2 ] Το σύστημα εμφανίζει φόρμα απογραφής με όλα τα προϊόντα και τις θεωρητικές ποσότητες
          \item [12.3 ] Ο χρήστης καταχωρεί τις πραγματικές ποσότητες που καταμέτρησε
          \item [12.4 ] Ο χρήστης επιλέγει την ακύρωση της απογραφής
          \item [12.5 ] Το σύστημα γυρίζει τον χρήστη στο βήμα 2 της βασικής ροής
        \end{enumerate}
  \item[13 ] Ακύρωση εκκρεμών παραγγελιών:
        \begin{enumerate}
          \item [13.1 ] Ο χρήστης επιλέγει την καρτέλα "Orders"
          \item [13.2 ] Ο χρήστης επιλέγει το "Pending Orders"
          \item [13.3 ] Το σύστημα εμφανίζει λίστα εκκρεμών παραγγελιών
          \item [13.4 ] Ο χρήστης επίλεγει ακύρωση διαδικασίας
          \item [13.5 ] Το σύστημα γυρίζει τον χρήστη στο βήμα 2 της βασικής ροής
        \end{enumerate}
  \item[14 ] Ακύρωση ληξιπρόθεσμων προϊόντων:
        \begin{enumerate}
          \item [14.1 ] Ο χρήστης επιλέγει την επιλογή "Expiration Check"
          \item [14.2 ] Το σύστημα εμφανίζει προϊόντα που πλησιάζουν ή έχουν ξεπεράσει την ημερομηνία λήξης
          \item [14.3 ] Ο χρήστης επιλέγει ακύρωση διαδικασίας
          \item [14.4 ] Το σύστημα γυρίζει τον χρήστη στο βήμα 2 της βασικής ροής
        \end{enumerate}
  \item[15 ] Ακύρωση εξαγωγής αναφορών:
        \begin{enumerate}
          \item [15.1 ] Ο χρήστης επιλέγει την επιλογή "Reports"
          \item [15.2 ] Το σύστημα εμφανίζει τους διαθέσιμους τύπους αναφορών %(τρέχον απόθεμα, ιστορικό κινήσεων, αξία αποθέματος, κ.λπ.)
          \item [15.3 ] Ο χρήστης επιλέγει ακύρωση διαδικασίας
          \item [15.4 ] Το σύστημα γυρίζει τον χρήστη στο βήμα 2 της βασικής ροής
        \end{enumerate}
  \item[16 ] Ακύρωση ακύρωσης παραγγελίας:
        \begin{enumerate}
          \item [16.1 ] Ο χρήστης επιλέγει την καρτέλα "Orders"
          \item [16.2 ] Ο χρήστης επιλέγει την παραγγελία που θέλει να ακυρώσει
          \item [16.3 ] Το σύστημα ζητά επιβεβαίωση για την ακύρωση
          \item [16.4 ] Ο χρήστης απορρίπτει την ακύρωση
          \item [16.5 ] Το σύστημα γυρίζει τον χρήστη στο βήμα 2 της βασικής ροής
        \end{enumerate}
\end{enumerate}

\subsection{Ειδικές απαιτήσεις}
\begin{itemize}
  \item Το σύστημα πρέπει να υποστηρίζει διαχείριση παρτίδων για προϊόντα με ημερομηνίες λήξης
  \item Απαιτείται καταγραφή όλων των κινήσεων αποθέματος (ιχνηλασιμότητα)
  \item Για φαρμακευτικά προϊόντα απαιτείται καταγραφή αριθμού παρτίδας και πλήρες ιστορικό διάθεσης
  \item Αυτοματοποιημένες ειδοποιήσεις για χαμηλό απόθεμα και προϊόντα που πλησιάζουν σε λήξη
  \item Υποστήριξη γραμμωτού κώδικα (barcode) και QR για γρήγορη αναγνώριση προϊόντων
  \item Υποστήριξη διαφορετικών μονάδων μέτρησης (τεμάχια, κουτιά, ml, γραμμάρια κτλ.)
  \item Δυνατότητα καταγραφής θέσης αποθήκευσης για κάθε προϊόν
\end{itemize}

\subsubsection{Μη λειτουργικές απαιτήσεις}
\begin{itemize}
  \item Ο χρόνος απόκρισης για αναζήτηση και ενημέρωση προϊόντων πρέπει να είναι λιγότερος από 1 δευτερόλεπτο
  \item Το σύστημα πρέπει να υποστηρίζει ταυτόχρονη πρόσβαση πολλαπλών χρηστών χωρίς συγκρούσεις δεδομένων
  \item Απαιτείται καθημερινό αυτόματο αντίγραφο ασφαλείας της βάσης δεδομένων αποθέματος
  \item Το σύστημα πρέπει να είναι διαθέσιμο 24/7 με διαθεσιμότητα τουλάχιστον 99.5%
  \item Η διεπαφή χρήστη πρέπει να είναι εύχρηστη και να επιτρέπει γρήγορες καταχωρήσεις ακόμα και σε συνθήκες πίεσης
\end{itemize}

\subsubsection{Περιβάλλον}
\begin{itemize}
  \item Διαδικτυακή πλατφόρμα προσβάσιμη μέσω φυλλομετρητών
  \item Εφαρμογή για κινητές συσκευές με δυνατότητα σάρωσης barcodes
  \item Διασύνδεση με συστήματα προμηθευτών για αυτόματη ενημέρωση τιμών και διαθεσιμότητας
  \item Υποστήριξη φορητών σαρωτών barcode και ασύρματων τερματικών για απογραφή
\end{itemize}

\subsection{Κατάσταση εισόδου}
\begin{itemize}
  \item Ο χρήστης έχει συνδεθεί στο σύστημα με τα κατάλληλα διαπιστευτήρια
  \item Ο χρήστης έχει τα απαραίτητα δικαιώματα για διαχείριση αποθήκης
  \item Η βάση δεδομένων αποθέματος είναι προσβάσιμη και λειτουργική
\end{itemize}

\subsection{Κάτασταση εξόδου}
\begin{itemize}
  \item Το απόθεμα έχει ενημερωθεί επιτυχώς
  \item Όλες οι κινήσεις έχουν καταγραφεί στο ιστορικό συναλλαγών
  \item Έχουν δημιουργηθεί οι αντίστοιχες ειδοποιήσεις για χαμηλό απόθεμα ή επικείμενες λήξεις
  \item Ο χρήστης επιστρέφει στην κεντρική οθόνη διαχείρισης αποθήκης
\end{itemize}

\subsection{Σημεία επέκτασης}
% \begin{itemize}
%   \item Προσθήκη συστήματος πρόβλεψης αναγκών βάσει ιστορικού κατανάλωσης και εποχικότητας
%   \item Διασύνδεση με λογιστικό σύστημα για αυτόματη ενημέρωση αξίας αποθέματος
%   \item Ενσωμάτωση RFID για αυτόματη παρακολούθηση αποθέματος σε πραγματικό χρόνο
%   \item Υποστήριξη διαχείρισης πολλαπλών αποθηκών/τοποθεσιών
%   \item Σύστημα προτεινόμενης αντικατάστασης προϊόντων όταν κάποιο δεν είναι διαθέσιμο
% \end{itemize}

\end{document} % chktex 17

% \section{Customer management}

% \subsection{Σύντομη Περιγραφή}

% \subsection{Χειριστές}

% \subsection{Γεγονός Έναρξης}

% \subsection{Ροή γεγονότων}

% \subsubsection{Βασική Ροή}

% \subsubsection{Εναλλακτικές Ροές}

% \subsection{Ειδικές απαιτήσεις}

% \subsubsection{Μη λειτουργικές απαιτήσεις}

% \subsubsection{Περιβάλλον}

% \subsection{Κατάσταση εισόδου}

% \subsection{Κάτασταση εξόδου}

% \subsection{Σημεία επέκτασης}