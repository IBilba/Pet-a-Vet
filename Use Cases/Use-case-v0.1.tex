\documentclass[12pt,a4paper,twoside]{book}
% Language setup for Greek and English
\usepackage[english,greek]{babel}
\usepackage[utf8]{inputenc}
\usepackage[T1,LGR]{fontenc}
% Font setup
\usepackage{fontspec}
\setmainfont{Linux Libertine O}[Scale=0.9]
\setsansfont{Linux Biolinum}[Scale=0.9]
\setmonofont{DejaVu Sans Mono}[Scale=0.9]
% Other necessary packages
\usepackage{enumitem}
\usepackage{graphicx}
\usepackage{amsmath}
\usepackage{makeidx}
\usepackage{multicol}
\usepackage{multirow}
\usepackage{hanging}
\usepackage{adjustbox}
\usepackage{amssymb}
\usepackage{stackengine}
\usepackage{ifthen}
\usepackage{array}
\usepackage{tcolorbox}
\tcbuselibrary{skins,breakable}
\usepackage{minted}
\usepackage{listings}
\usepackage{parskip}
\usepackage{float}
\usepackage{subcaption}
\usepackage{xcolor}
\usepackage{cancel}
\usepackage{pdfpages}
\usepackage{titlesec}
% Page & text layout
\usepackage{geometry}
\geometry{
   a4paper,
   top=2.5cm,
   bottom=2.5cm,
   left=2.5cm,
   right=2.5cm
}
% Headers & footers
\usepackage{fancyhdr}
\pagestyle{fancyplain}
\fancyhf{} % Clear all header/footer fields

% Define plain style for chapter pages and TOC
\fancypagestyle{plain}{%
    \fancyhf{} % Clear all header/footer fields
    \renewcommand{\headrulewidth}{0pt} % Remove header rule
    \renewcommand{\footrulewidth}{0pt} % Remove footer rule
}

% Define TOC style with specific headers
\fancypagestyle{tocstyle}{%
    \fancyhf{} % Clear all header/footer fields
    \fancyhead[LE]{\thepage}
    \fancyhead[CE]{\leftmark}
    \fancyhead[CO]{\rightmark}
    \fancyhead[RO]{\thepage}
    \renewcommand{\headrulewidth}{0pt}
}

% Redefine \tableofcontents to use tocstyle
\let\oldtableofcontents\tableofcontents
\renewcommand{\tableofcontents}{%
    \clearpage
    % First page of TOC should be empty
    \thispagestyle{empty}
    \pagestyle{tocstyle}
    \oldtableofcontents % chktex 1
    \clearpage
    \pagestyle{fancy}
    % Regular pages header settings
    \fancyhead[LE]{\thepage} % Left header on Even pages: Shows page number
    \fancyhead[CE]{\leftmark} % Center header on Even pages: Shows chapter name (stored in \leftmark)
    \fancyhead[RE]{ΚΕΦ. \thechapter} % Right header on Even pages: Shows "ΚΕΦ." followed by chapter number % chktex 12

    \fancyhead[LO]{\thesection} % Left header on Odd pages: Shows current section number
    \fancyhead[CO]{\rightmark} % Center header on Odd pages: Shows section name (stored in \rightmark)
    \fancyhead[RO]{\thepage} % Right header on Odd pages: Shows page number

    \renewcommand{\headrulewidth}{0.4pt} % Sets thickness of the horizontal line below the header

    % Redefine chapter mark to remove "Chapter" prefix
    \renewcommand{\chaptermark}[1]{\markboth{#1}{}}
    % Redefine section mark to use section title only
    \renewcommand{\sectionmark}[1]{\markright{#1}}
}

% Make table of contents use plain style
\addtocontents{toc}{\protect\thispagestyle{plain}}

% Hyperlinks
\usepackage{hyperref}
\usepackage{bookmark}                                       % Add this line to fix the hyperref warnings
\hypersetup{
    colorlinks=true,
    linkcolor=blue,
    citecolor=blue,
    filecolor=blue,
    urlcolor=cyan,
    unicode,
    pdftitle={1ο Project Λειτουργικών Συστημάτων},
    pdfsubject={Αναφορά Εργασίας},
    pdfborderstyle={/S/U/W 1},                              % Underline links
}
\usepackage{tikz}
\usetikzlibrary{positioning, shapes.geometric, arrows.meta}
\usepackage{pgfplots}
\pgfplotsset{compat=1.18}
\usetikzlibrary{calc,arrows.meta,positioning}

\definecolor{customRed}{HTML}{FF5F5A}                       % using hexadecimal
\definecolor{customYellow}{HTML}{FFBE2E}                    % using hexadecimal
\definecolor{customGreen}{HTML}{2ACA44}                     % using hexadecimal
\definecolor{customPurple}{HTML}{C477DB}                    % using hexadecimal
\definecolor{customBlue}{HTML}{052538}                      % using hexadecimal
\definecolor{customLightBlue}{HTML}{60ADEC}                 % using hexadecimal
\definecolor{customGray}{HTML}{A0A0A0}                      % using hexadecimal
\definecolor{customBlack}{HTML}{000000}                     % using hexadecimal

% Define a lighter gray for background and darker colors for text
\definecolor{customBgGray}{HTML}{E0E0E0}         % Light gray background
\definecolor{customCodeRed}{HTML}{C92A2A}        % Darker red
\definecolor{customCodeYellow}{HTML}{E67700}     % Darker yellow/orange
\definecolor{customCodeGreen}{HTML}{087F23}      % Darker green
\definecolor{customCodePurple}{HTML}{7B1FA2}     % Darker purple
\definecolor{customCodeBlue}{HTML}{1976D2}       % Darker blue
\definecolor{customCodeBlack}{HTML}{1A1A1A}      % Lighter black

\lstdefinestyle{myPythonStyle}{
    language=Python,                                            % Set the language to Python
    backgroundcolor=\color{customBgGray},                       % Background color
    basicstyle=\color{customCodeBlack}\ttfamily\footnotesize,   % Basic font style, size, and color
    keywordstyle=\color{customCodePurple}\bfseries,             % Style for general keywords
    keywordstyle=[2]\color{customCodeYellow}\bfseries,          % Style for specific keywords
    commentstyle=\color{customCodeGreen}\itshape,               % Style for comments
    stringstyle=\color{customCodeGreen},                        % Style for strings
    identifierstyle=\color{customCodeBlue},                     % Style for identifiers
    morekeywords={self, None, True, False, format, abs,
                as, pass, return, if, elif, else, for,
                range, while, try, except, with, lambda,
                yield, global, nonlocal, assert, del,
                raise, in, is, and, or, not},                   % Additional keywords
    morekeywords=[2]{os, graphviz, Digraph, collections,
                    deque, math, tabulate},                     % Define 'import' and 'from' as additional keywords
    % procnamekeys={def,class},                                 % Highlight function and class names
    showstringspaces=false,                                     % Don't show spaces in strings
%
    breaklines=true,                                            % Automatically break long lines
    breakatwhitespace=false,                                    % Break lines at any character
    % prebreak=\textbackslash,                                  % Character for breaking lines
    postbreak={\space},                                         % Character for breaking lines
    breakautoindent=false,                                      % Indentation after line break
    breakindent=0pt,                                            % Indentation before line break
    resetmargins=true,                                          % Reset margins after line break
    keepspaces=true,                                            % Keep spaces in the code
    showspaces=false,                                           % Don't show spaces
    columns=flexible,                                           % Column format
%
    numbers=left,                                               % Line numbers on the left
    numberstyle=\small\color{customBgGray},                     % Style for line numbers
    % numwidth=4em,                                             % Width allocated for line numbers
    stepnumber=1,                                               % Step between line numbers
    numbersep=8pt,                                              % Distance of line numbers from code
    xleftmargin=1.5em,                                          % Match the numwidth
    tabsize=4,                                                  % Size of a tab
    captionpos=t,                                               % Position of the caption (t for top)
    firstnumber=auto,                                           % Continue line numbering from previous listing
    aboveskip=0em,                                              % Remove external spacing
    belowskip=0em,                                              % Remove external spacing
    frame=none,                                                 % Add top and bottom rules only
    framerule=0pt,                                              % Make the rules invisible
    rulecolor=\color{customBgGray},                             % Color of the frame (if enabled)
    framesep=2pt,                                               % Add internal padding
    % framexleftmargin=1em,                                     % Add internal left margin
    % framexrightmargin=1em,                                    % Add internal right margin
    escapeinside={\(*@}{@*\)},                                  % For escaping characters
    morecomment=[l]{\#},                                        % Define comment style
    morestring=[b]',                                            % Define string style with single quotes
    morestring=[b]",                                            % Define string style with double quotes % chktex 18
    literate={
        {``}{{\textquotedbleft}}2
        {''}{{\textquotedbright}}2
        {`}{{\textquoteleft}}1
        {'}{{\textquoteright}}1
        {_}{{\_}}1
    },
}

\lstdefinestyle{myCStyle}{
    language=C,
    backgroundcolor=\color{customBgGray},                       % Background color
    basicstyle=\color{customCodeBlack}\ttfamily\footnotesize,   % Basic font style
    keywordstyle=\color{customCodePurple}\bfseries,            % Style for keywords
    keywordstyle=[2]\color{customCodeYellow}\bfseries,         % Style for additional keywords
    commentstyle=\color{customCodeGreen}\itshape,              % Style for comments
    stringstyle=\color{customCodeGreen},                       % Style for strings
    identifierstyle=\color{customCodeBlue},                    % Style for identifiers
    morekeywords={void, int, char, float, double, long, short, signed, unsigned,
                  const, static, extern, volatile, register, auto, struct, union,
                  typedef, enum, sizeof, break, continue, goto, return, if, else,
                  switch, case, default, for, while, do, pid_t, sem_t},
    morekeywords=[2]{stdio.h, stdlib.h, string.h, math.h, time.h, ctype.h,
                     unistd.h, semaphore.h, sys/mman.h, sys/wait.h, fcntl.h,
                     printf, fprintf, perror, exit, malloc, free, fork, waitpid,
                     mmap, munmap, sem_init, sem_wait, sem_post, sem_destroy,
                     sleep, rand, srand, atoi, time, MAP_SHARED, MAP_ANONYMOUS,
                     PROT_READ, PROT_WRITE, MAP_FAILED, EXIT_SUCCESS, EXIT_FAILURE},
    showstringspaces=false,
    frame=none,
    rulecolor=\color{customBgGray},
    breaklines=true,                % Break long lines
    breakatwhitespace=false,        % Break at any character
    postbreak={\space},             % Character after break
    breakautoindent=false,          % No indentation after break
    breakindent=0pt,                % No indent before break
    resetmargins=true,              % Reset margins after break
    keepspaces=true,                % Preserve spaces
    showspaces=false,               % Don't show spaces
    columns=flexible,               % Column format
    numbers=left,
    numberstyle=\small\color{customBgGray},
    stepnumber=1,
    numbersep=8pt,
    xleftmargin=1.5em,
    tabsize=4,
    captionpos=t,
    firstnumber=auto,
    aboveskip=0em,
    belowskip=0em,
    framesep=2pt,
    escapeinside={\(*@}{@*\)},
    morecomment=[l]{//},
    morecomment=[s]{/*}{*/},
    morestring=[b]", % chktex 18
    morestring=[b]',
    literate={
        {``}{{\textquotedbleft}}2
        {''}{{\textquotedbright}}2
        {`}{{\textquoteleft}}1
        {'}{{\textquoteright}}1
        {_}{{\_}}1
    },
}

\lstdefinestyle{myBashStyle}{
    language=bash,
    backgroundcolor=\color{customBgGray},                       % Background color
    basicstyle=\color{customCodeBlack}\ttfamily\footnotesize,   % Basic font style
    keywordstyle=\color{customCodePurple}\bfseries,            % Style for keywords
    keywordstyle=[2]\color{customCodeYellow}\bfseries,         % Style for additional keywords
    commentstyle=\color{customCodeGreen}\itshape,              % Style for comments
    stringstyle=\color{customCodeGreen},                       % Style for strings
    identifierstyle=\color{customCodeBlue},                    % Style for identifiers
    alsoletter={0123456789}                                    % Define numbers as letters
    morekeywords={if, then, else, elif, fi, for, while, do, done, in, case, esac, 
                  function, select, until, break, continue, return, exit, shift,
                  declare, local, readonly, export, set, unset},
    morekeywords=[2]{echo, read, cat, awk, grep, sed, cut, tr, sort, uniq, wc,
                     mkdir, rm, cp, mv, ls, cd, pwd, touch, chmod, chown, find,
                     test, source, alias, eval, exec, getopts, printf, wait,
                     STDIN, STDOUT, STDERR, IFS, PATH, BEGIN, END, NR, NF, 
                     tolower, concat, print},
    showstringspaces=false,
    frame=none,
    rulecolor=\color{customBgGray},
    breaklines=true,                % Break long lines
    breakatwhitespace=false,        % Break at any character
    postbreak={\space},             % Character after break
    breakautoindent=false,          % No indentation after break
    breakindent=0pt,                % No indent before break
    resetmargins=true,              % Reset margins after break
    keepspaces=true,                % Preserve spaces
    showspaces=false,               % Don't show spaces
    columns=flexible,               % Column format
    numbers=left,
    numberstyle=\small\color{customBgGray},
    stepnumber=1,
    numbersep=8pt,
    xleftmargin=1.5em,
    tabsize=4,
    captionpos=t,
    firstnumber=auto,
    aboveskip=0em,
    belowskip=0em,
    framesep=2pt,
    escapeinside={\(*@}{@*\)},
    morecomment=[l]{\#},
    morestring=[b]", % chktex 18
    morestring=[b]',
    literate={
        {``}{{\textquotedbleft}}2
        {''}{{\textquotedbright}}2
        {`}{{\textquoteleft}}1
        {'}{{\textquoteright}}1
        {_}{{\_}}1
    },
}

\tcbset{
    myCustomStyle/.style={
        enhanced,                       % Enable enhanced features
        breakable,                      % Allow the box to break across pages
        colback=customBlue,             % Background color
        colframe=customBgGray,              % Frame color
        listing only,                   % Use listings
%
        fonttitle=\bfseries,            % Title font style
        % frame=single,                 % Frame type
        % interior titled,              % Use regular titled interior
        interior style={
            top color=black!75,         % Title area background
            bottom color=black!75,      % Title area background
        },
        colbacktitle=black!75,          % Title background color
        coltitle=white,                 % Title color
        rounded corners,                % Rounded corners
        boxrule=1mm,                    % Frame thickness
        drop shadow southeast,          % Shadow effect
        lefttitle=0pt,                  % Remove left margin of title
        % left=1em,                     % Increase left margin for line numbers
        title={\hspace*{-1em}\textcolor{customRed}{● }\textcolor{customYellow}{● }\textcolor{customGreen}{●}\quad#1}, % Circles + Title
        attach title to upper={\vspace{2.3mm}}, % Adjust vertical spacing
%
        beforeafter skip=8pt,           % Space before and after the box
        breaklines=true,                % Allow the box to break across pages
        breakatwhitespace=false,        % Break lines at any character if necessary
        % sharp corners                 % Sharp corners for the title area
    }
}

% Custom commands for language switching
\newcommand{\en}[1]{\foreignlanguage{english}{#1}}
\newcommand{\gr}[1]{\foreignlanguage{greek}{#1}}
% Custom command for coloring
\newcommand{\blue}[1]{\textcolor{blue}{#1}}
% Index
\makeindex
% Set headheight to avoid fancyhdr warnings
\setlength{\headheight}{14.5pt}

% Adjust chapter spacing
\titleformat{\chapter}[display]
{\normalfont\huge\bfseries}{\chaptertitlename\ \thechapter}{15pt}{\Huge}
\titlespacing*{\chapter}{0pt}{-10pt}{20pt}

% Adjust section spacing
\titlespacing*{\section}{0pt}{10pt}{10pt}

% Adjust vertical space between paragraphs
\setlength{\parskip}{0.5em}

% Make figure placement less strict
\setcounter{topnumber}{2}
\setcounter{bottomnumber}{2}
\setcounter{totalnumber}{4}
\renewcommand{\topfraction}{0.85}
\renewcommand{\bottomfraction}{0.85}
\renewcommand{\textfraction}{0.15}
\renewcommand{\floatpagefraction}{0.7}

\begin{document}
% Start Arabic numbering from the beginning but hide numbers
\pagenumbering{arabic}
% Add this command to prevent blank pages
\let\cleardoublepage\clearpage

% Set main language to Greek
\selectlanguage{greek}

% Title page with hidden number
\begin{titlepage}
  \vspace*{8cm}
  \begin{center}
    {\LARGE \textbf{\textit{Pet-à-Vet}}}\\ % chktex 8
    \vspace*{0.5cm}
    {\large Use Cases}\\
    {\large Τεχνολογία Λογισμικού}
    \vspace*{5cm}
    \\
    \vspace*{1cm}
    \begin{tabular}{l l l}
      % \large Βασίλειος Μπίτζας            & \large 1083796 & \large up1083796@ac.upatras.gr \\
      % \large Αριάδνη Σβωλοπούλου          & \large 1104806 & \large up1104806@ac.upatras.gr \\
    \end{tabular}

    \vspace*{1cm}

    {\normalsize Τμήμα Μηχανικών Η/Υ \& Πληροφορικής}
    \\
    {\normalsize Πανεπιστήμιο Πατρών}
    \\
  \end{center}
  \thispagestyle{empty} % Suppress page number on the title page
\end{titlepage}

% % Add this command to prevent blank pages
% \let\cleardoublepage\clearpage
% \frontmatter
\tableofcontents
% \mainmatter % chktex 1

\printindex

\chapter{Use Cases}

\section{Customer management}

\subsection{Σύντομη Περιγραφή}
Η περίπτωση χρήσης "Customer Management" επιτρέπει στους διαχειριστές της κτηνιατρικής κλινικής και στο προσωπικό υποδοχής να διαχειρίζονται πλήρως τα προφίλ των πελατών (ιδιοκτητών κατοικίδιων). Αυτό περιλαμβάνει την καταχώρηση νέων πελατών, την ενημέρωση των στοιχείων τους, την προβολή του ιστορικού επισκέψεων, την αντιστοίχιση με τα κατοικίδιά τους, και την παρακολούθηση οικονομικών συναλλαγών. % chktex 18

\subsection{Χειριστές}
\begin{itemize}
  \item Διαχειριστής συστήματος
  \item Κτηνίατρος
  \item Γραμματέας
\end{itemize}

\subsection{Γεγονός Έναρξης}
Η περίπτωση χρήσης ξεκινά όταν:
\begin{itemize}
  \item Ένας διαχειριστής, κτηνίατρος ή υπάλληλος υποδοχής επιλέγει την επιλογή "Customer Management" από το κεντρικό μενού % chktex 18
        %   \item Ένας νέος πελάτης εγγράφεται στο σύστημα μέσω της διαδικτυακής πλατφόρμας
        %   \item Το προσωπικό χρειάζεται να αναζητήσει πληροφορίες για έναν πελάτη
        %   \item Ο ιδιοκτήτης κατοικίδιου συνδέεται στο λογαριασμό του για να επεξεργαστεί τα στοιχεία του
\end{itemize}

\subsection{Ροή γεγονότων}

\subsubsection{Βασική Ροή}
\begin{enumerate}
  \item Ο χρήστης μπαίνει στην ενότητα "Customer Management". % chktex 18
  \item Το σύστημα εμφανίζει μια λίστα με τους υπάρχοντες πελάτες.
  \item Ο χρήστης επιλέγει τη προσθήκη πελάτη.
  \item Το σύστημα εμφανίζει φόρμα με τα απαραίτητα πεδία %(ονοματεπώνυμο, στοιχεία επικοινωνίας, διεύθυνση κτλ.).
  \item Ο χρήστης συμπληρώνει τα στοιχεία και υποβάλλει τη φόρμα.
  \item Το σύστημα επικυρώνει τα δεδομένα, δημιουργεί και αποθηκεύει το νέο προφίλ πελάτη.
  \item Το σύστημα αποστέλλει αυτόματα email/SMS καλωσορίσματος με οδηγίες ενεργοποίησης λογαριασμού.
  \item Το σύστημα οδηγεί τον χρήστη στη βήμα 2 της βασικής ροής.
\end{enumerate}

\subsubsection{Εναλλακτικές Ροές}
\begin{enumerate}
  \item[1 ] "Search"(/προβολή) υπάρχοντος πελάτη: % chktex 18 chktex 36
        \begin{enumerate}
          \item[1.1 ] Ο χρήστης επιλέγει την επιλογή αναζήτησης
          \item[1.2 ] Ο χρήστης εισάγει κριτήρια αναζήτησης %(όνομα, email, τηλέφωνο, κ.λπ.)
          \item[1.3 ] Το σύστημα εμφανίζει τα αποτελέσματα που ταιριάζουν
          \item[1.4 ] Ο χρήστης επιλέγει την προβολή του επιθυμητού πελάτη από τα αποτελέσματα και προβάλει τα στοχεία του
          \item[1.5 ] Ο χρήστης κλείνει το παράθυρο αναζήτησης και επιστρέφει στο βήμα 2 της βασικής ροής
        \end{enumerate}
  \item[2 ] Επεξεργασία προφίλ πελάτη:
        \begin{enumerate}
          \item [2.1 ] Ο χρήστης επιλέγει την επιλογή αναζήτησης
          \item [2.2 ] Ο χρήστης εισάγει κριτήρια αναζήτησης %(όνομα, email, τηλέφωνο, κ.λπ.)
          \item [2.3 ] Το σύστημα εμφανίζει τα αποτελέσματα που ταιριάζουν
          \item [2.4 ] Ο χρήστης επιλέγει την επεξεργασία του επιθυμητού πελάτη από τα αποτελέσματα και επεξεργάζεται τα στοιχεία του
          \item [2.5 ] Ο χρήστης αποθηκεύει τις αλλαγές και επιστρέφει στο βήμα 2 της βασικής ροής
        \end{enumerate}
  \item[3 ] "Sort" πελατών: % chktex 18
        \begin{enumerate}
          \item [3.1 ] Ο χρήστης επιλέγει την επιλογή ταξινόμησης
          \item [3.2 ] Ο χρήστης επιλέγει το κριτήριο ταξινόμησης %(όνομα, ημερομηνία εγγραφής, κ.λπ.)
          \item [3.3 ] Το σύστημα ταξινομεί τους πελάτες σύμφωνα με το επιλεγμένο κριτήριο και τους εμφανίζει στην οθόνη
        \end{enumerate}
  \item[4 ] "Filter" πελατών: % chktex 18
        \begin{enumerate}
          \item [4.1 ] Ο χρήστης επιλέγει την επιλογή φιλτραρίσματος
          \item [4.2 ] Ο χρήστης επιλέγει το κριτήριο φιλτραρίσματος %(κατάσταση λογαριασμού, ημερομηνία εγγραφής, κατοικίδιο, κ.λπ.)
          \item [4.3 ] Ο χρήστης εισάγει τις τιμές του κριτηρίου φιλτραρίσματος
          \item [4.4 ] Το σύστημα εφαρμόζει το φίλτρο στη λίστα των πελατών και τους εμφανίζει στην οθόνη
        \end{enumerate}
  \item[5 ] Λανθασμένη εισαγωγή δεδομένων:
        \begin{enumerate}
          \item [5.1 ] Ο χρήστης επιλέγει την επιλογή προσθήκης
          \item [5.2 ] Ο χρήστης εισάγει τα στοιχεία του νέου πελάτη σε λανθασμένη μορφή
          \item [5.3 ] Το σύστημα εμφανίζει μήνυμα λάθους και οδηγεί τον χρήστη στο βήμα 4 της βασικής ροής
        \end{enumerate}
  \item[6 ] Ακύρωση προσθήκης:
        \begin{enumerate}
          \item [6.1 ] Ο χρήστης επιλέγει την επιλογή προσθήκης
          \item [6.2 ] Ο χρήστης εισάγει τα στοιχεία του νέου πελάτη
          \item [6.3 ] Ο χρήστης επιλέγει την επιλογή ακύρωσης
          \item [6.4 ] Το σύστημα εμφανίζει μήνυμα επιβεβαίωσης και οδηγεί τον χρήστη στο βήμα 2 της βασικής ροής
        \end{enumerate}
  \item[7 ] Ακύρωση επεξεργασίας:
        \begin{enumerate}
          \item [7.1 ] Ο χρήστης επιλέγει την επιλογή αναζήτησης
          \item [7.2 ] Ο χρήστης εισάγει κριτήρια αναζήτησης %(όνομα, email, τηλέφωνο, κ.λπ.)
          \item [7.3 ] Το σύστημα εμφανίζει τα αποτελέσματα που ταιριάζουν
          \item [7.4 ] Ο χρήστης επιλέγει την επεξεργασία του επιθυμητού πελάτη από τα αποτελέσματα
          \item [7.5 ] Ο χρήστης επιλέγει την επιλογή ακύρωσης
          \item [7.6 ] Το σύστημα εμφανίζει μήνυμα επιβεβαίωσης και οδηγεί τον χρήστη στο βήμα 2 της βασικής ροής
        \end{enumerate}
  \item[8 ] Λανθασμένα κριτήρια αναζήτησης:
        \begin{enumerate}
          \item [8.1 ] Ο χρήστης επιλέγει την επιλογή αναζήτησης
          \item [8.2 ] Ο χρήστης εισάγει λανθασμένα κριτήρια αναζήτησης
          \item [8.3 ] Το σύστημα εμφανίζει μήνυμα λάθους και οδηγεί τον χρήστη στο βήμα 2 της εναλλακτικής ροής
        \end{enumerate}
  \item[9 ] Πάτημα πελάτη:
        \begin{enumerate}
          \item [9.1 ] Ο χρήστης επιλέγει έναν πελάτη από αυτούς που εμφανίζονται αρχικά στην οθόνη
          \item [9.2 ] Το σύστημα εμφανίζει τα στοιχεία του επιλεγμένου πελάτη
          \item [9.3 ] Ο χρήστης πατάει κλείσιμο και το σύστημα τον γυρνάει στο βήμα 2 της βασικής ροής
        \end{enumerate}
        %   \item[10 ] Εξαγωγή στατιστικών:
        %     \begin{enumerate}
        %         \item [10.1 ] Ο χρήστης επιλέγει την επιλογή εξαγωγής
        %         \item [10.2 ] Ο χρήστης επιλέγει το είδος των στατιστικών που επιθυμεί (πλήθος πελατών, κατανομή κατοικιδίων, κ.λπ.)
        %         \item [10.3 ] Το σύστημα εμφανίζει τα στατιστικά στοιχεία στην οθόνη και παρέχει τη δυνατότητα εξαγωγής σε αρχείο
        %     \end{enumerate}
\end{enumerate}

\subsection{Ειδικές απαιτήσεις}
\begin{itemize}
  \item Το σύστημα πρέπει να συμμορφώνεται με τους κανονισμούς GDPR για την προστασία προσωπικών δεδομένων
  \item Τα δεδομένα των πελατών θα πρέπει να κρυπτογραφούνται κατά την αποθήκευση
  \item Η πρόσβαση στα οικονομικά στοιχεία περιορίζεται σε χρήστες με ειδική εξουσιοδότηση
  \item Το σύστημα πρέπει να διατηρεί αρχείο καταγραφής όλων των τροποποιήσεων (audit trail)
  \item Υποστήριξη μαζικής εισαγωγής πελατών μέσω αρχείου CSV
\end{itemize}

\subsubsection{Μη λειτουργικές απαιτήσεις}
\begin{itemize}
  \item Ο χρόνος απόκρισης για αναζήτηση πελατών πρέπει να είναι λιγότερος από 2 δευτερόλεπτα ακόμα και με βάση δεδομένων 10.000+ πελατών
  \item Το σύστημα πρέπει να παραμένει λειτουργικό ακόμα και με περιορισμένη συνδεσιμότητα στο διαδίκτυο, διατηρώντας τοπικό αντίγραφο των πιο πρόσφατων συναλλαγών
  \item Το περιβάλλον χρήσης πρέπει να είναι προσβάσιμο από άτομα με αναπηρίες (WCAG 2.1 επίπεδο AA)
  \item Το σύστημα πρέπει να υποστηρίζει πολλαπλές γλώσσες (Ελληνικά και Αγγλικά αρχικά)
\end{itemize}

\subsubsection{Περιβάλλον}
\begin{itemize}
  \item Διαδικτυακή πλατφόρμα προσβάσιμη μέσω φυλλομετρητών (Chrome, Firefox, Safari, Edge)
  \item Εφαρμογή για κινητές συσκευές (iOS και Android)
        %   \item Διασύνδεση με συστήματα τιμολόγησης και λογιστικά προγράμματα
        %   \item Εκτυπωτές για αποδείξεις και έγγραφα
\end{itemize}

\subsection{Κατάσταση εισόδου}
\begin{itemize}
  \item Ο χρήστης έχει συνδεθεί στο σύστημα με τα κατάλληλα διαπιστευτήρια
  \item Η βάση δεδομένων του συστήματος είναι προσβάσιμη και λειτουργική
\end{itemize}

\subsection{Κάτασταση εξόδου}
\begin{itemize}
  \item Τα στοιχεία του πελάτη έχουν καταχωρηθεί ή ενημερωθεί επιτυχώς στη βάση δεδομένων
  \item Όλες οι ενέργειες έχουν καταγραφεί στο αρχείο ιστορικού του συστήματος
  \item Έχουν σταλεί οι κατάλληλες ειδοποιήσεις στους εμπλεκόμενους (π.χ. email επιβεβαίωσης στον πελάτη)  % chktex 12
  \item Ο χρήστης επιστρέφει στην κεντρική οθόνη διαχείρισης πελατών % ή προχωρά σε σχετική λειτουργία (π.χ. προγραμματισμός ραντεβού)
\end{itemize}

\subsection{Σημεία επέκτασης}
% \begin{itemize}
%     \item Διασύνδεση με σύστημα αυτόματης αποστολής υπενθυμίσεων για εμβολιασμούς και προληπτικούς ελέγχους
%     \item Ενσωμάτωση συστήματος πιστότητας (loyalty) με πόντους και εκπτώσεις
%     \item Ανάπτυξη εξατομικευμένων προτάσεων υπηρεσιών και προϊόντων βάσει του προφίλ του πελάτη
%     \item Δυνατότητα ψηφιακής υπογραφής εγγράφων συγκατάθεσης για θεραπείες
%     \item Ενσωμάτωση τεχνολογίας αναγνώρισης προσώπου για γρήγορη ταυτοποίηση πελατών στην υποδοχή
% \end{itemize}

\section{Warehouse management}

\subsection{Σύντομη Περιγραφή}
Η περίπτωση χρήσης "Warehouse Management" αποτελεί την κεντρική λειτουργία διαχείρισης αποθήκης της κτηνιατρικής κλινικής. Περιλαμβάνει έξι βασικές υπο-περιπτώσεις χρήσης: % chktex 18
\begin{itemize}
  \item Manage Products --- Διαχείριση προϊόντων και καταλόγου
  \item Perform Inventory --- Εκτέλεση απογραφών αποθέματος
  \item Manage Orders --- Διαχείριση παραγγελιών προς προμηθευτές
  \item Search Products --- Αναζήτηση και εύρεση προϊόντων
  \item Generate Reports --- Δημιουργία αναφορών αποθήκης
  \item Monitor Expiration --- Παρακολούθηση ημερομηνιών λήξης
\end{itemize}

Μέσω αυτών των λειτουργιών, το σύστημα επιτρέπει την ολοκληρωμένη διαχείριση του αποθέματος, συμπεριλαμβανομένων φαρμάκων, αναλωσίμων, τροφών και προϊόντων περιποίησης.

\subsection{Χειριστές}
\begin{itemize}
  \item Διαχειριστής συστήματος
  \item Κτηνίατρος
  \item Γραμματέας
\end{itemize}

\subsection{Γεγονός Έναρξης}
Η περίπτωση χρήσης ξεκινά όταν:
\begin{itemize}
  \item Ένας εξουσιοδοτημένος χρήστης επιλέγει την ενότητα "Warehouse Management" % chktex 18
\end{itemize}

\subsection{Ροή γεγονότων}

\subsubsection{Βασική Ροή}
\begin{enumerate}
  \item Ο χρήστης εισέρχεται στο σύστημα διαχείρισης αποθήκης.
  \item Το σύστημα οδηγεί τον χρήστη στην καρτέλα "Manage Products". % chktex 18
  \item Το σύστημα εμφανίζει το κεντρικό μενού διαχείρισης αποθήκης.
\end{enumerate}

\subsubsection{Εναλλακτικές Ροές}
\begin{enumerate}
  % \item[1 ] Ειδοποιήσεις συστήματος:
  %       \begin{enumerate}
  %         \item[1.1 ] Το σύστημα εμφανίζει ειδοποιήσεις για:
  %               \begin{itemize}
  %                 \item Χαμηλά αποθέματα
  %                 \item Προϊόντα που πλησιάζουν σε λήξη
  %                 \item Εκκρεμείς παραγγελίες
  %                 \item Προγραμματισμένες απογραφές
  %               \end{itemize}
  %         \item[1.2 ] Ο χρήστης μπορεί να μεταβεί απευθείας στη σχετική λειτουργία
  %       \end{enumerate}
  % \item[2 ] Γρήγορη πρόσβαση:
  %       \begin{enumerate}
  %         \item[2.1 ] Ο χρήστης χρησιμοποιεί συντομεύσεις για συχνές λειτουργίες
  %         \item[2.2 ] Το σύστημα παρέχει άμεση πρόσβαση στην επιλεγμένη λειτουργία
  %       \end{enumerate}
\end{enumerate}

\subsection{Ειδικές απαιτήσεις}
\begin{itemize}
  \item Ενοποιημένη διεπαφή για όλες τις λειτουργίες διαχείρισης αποθήκης
  \item Κεντρικό σύστημα ειδοποιήσεων και επισημάνσεων
  \item Διασύνδεση μεταξύ των διαφορετικών υπο-συστημάτων
  \item Ολοκληρωμένο σύστημα διαχείρισης δικαιωμάτων πρόσβασης
\end{itemize}

\subsubsection{Μη λειτουργικές απαιτήσεις}
\begin{itemize}
  \item Γρήγορη εναλλαγή μεταξύ διαφορετικών λειτουργιών
  \item Υψηλή διαθεσιμότητα συστήματος (24/7 λειτουργία)
  \item Ασφαλής πρόσβαση και προστασία δεδομένων
  \item Δυνατότητα ταυτόχρονης χρήσης από πολλαπλούς χρήστες
\end{itemize}

\subsubsection{Περιβάλλον}
\begin{itemize}
  \item Ενιαία διαδικτυακή πλατφόρμα
  \item Υποστήριξη όλων των σύγχρονων φυλλομετρητών
  \item Προσαρμοστική διεπαφή για διάφορες συσκευές
\end{itemize}

\subsection{Κατάσταση εισόδου}
\begin{itemize}
  \item Ο χρήστης έχει συνδεθεί με έγκυρα διαπιστευτήρια
  \item Το σύστημα είναι διαθέσιμο και λειτουργικό
\end{itemize}

\subsection{Κάτασταση εξόδου}
\begin{itemize}
  \item Επιτυχής πρόσβαση στην επιλεγμένη λειτουργία
  \item Καταγραφή της δραστηριότητας στο σύστημα
\end{itemize}

\subsection{Σημεία επέκτασης}
\begin{itemize}
  \item Ενσωμάτωση νέων μονάδων διαχείρισης αποθήκης
  \item Προσθήκη προηγμένων αναλυτικών στοιχείων
  \item Διασύνδεση με εξωτερικά συστήματα προμηθευτών
  \item Επέκταση για διαχείριση πολλαπλών αποθηκών
\end{itemize}

\section{Manage Products}

\subsection{Σύντομη Περιγραφή}
Η περίπτωση χρήσης "Manage Products" επιτρέπει στους εξουσιοδοτημένους χρήστες να διαχειρίζονται τα προϊόντα της κτηνιατρικής κλινικής. Αυτό περιλαμβάνει την προσθήκη νέων προϊόντων, την επεξεργασία υπαρχόντων, την ενημέρωση των τιμών και των χαρακτηριστικών τους, καθώς και την κατηγοριοποίησή τους. Μέσω αυτής της περίπτωσης χρήσης, το προσωπικό μπορεί να διατηρεί έναν ενημερωμένο κατάλογο προϊόντων για τη σωστή λειτουργία της κλινικής. % chktex 18

\subsection{Χειριστές}
\begin{itemize}
  \item Διαχειριστής συστήματος
  \item Κτηνίατρος
  \item Γραμματέας
\end{itemize}

\subsection{Γεγονός Έναρξης}
Η περίπτωση χρήσης ξεκινά όταν:
\begin{itemize}
  \item Ένας εξουσιοδοτημένος χρήστης επιλέγει την επιλογή "Manage Products" από την ενότητα "Warehouse Management" ή από την απλή επιλογή της ενότητας "Warehouse Management" % chktex 18
\end{itemize}

\subsection{Ροή γεγονότων}

\subsubsection{Βασική Ροή}
\begin{enumerate}
  \item Ο χρήστης επιλέγει την ενότητα "Manage Products" από το μενού διαχείρισης αποθήκης. % chktex 18
  \item Το σύστημα εμφανίζει μια λίστα με τα υπάρχοντα προϊόντα.
  \item Ο χρήστης επιλέγει την προσθήκη νέου προϊόντος.
  \item Το σύστημα εμφανίζει φόρμα με τα απαραίτητα πεδία %(κωδικός, όνομα, κατηγορία, προμηθευτής, τιμή αγοράς, τιμή πώλησης, ελάχιστο απόθεμα, μονάδα μέτρησης, κτλ.).
  \item Ο χρήστης συμπληρώνει τα στοιχεία και υποβάλλει τη φόρμα.
  \item Το σύστημα επικυρώνει τα δεδομένα και αποθηκεύει το νέο προϊόν στη βάση δεδομένων.
  \item Το σύστημα επιβεβαιώνει την επιτυχή προσθήκη του προϊόντος.
  \item Το σύστημα επιστρέφει στη λίστα προϊόντων με το νέο προϊόν προστιθέμενο.
\end{enumerate}

\subsubsection{Εναλλακτικές Ροές}
\begin{enumerate}
  \item[1 ] Επεξεργασία υπάρχοντος προϊόντος:
        \begin{enumerate}
          \item[3.1.1 ] Ο χρήστης επιλέγει ένα προϊόν από τη λίστα για επεξεργασία
          \item[3.1.2 ] Το σύστημα εμφανίζει τη φόρμα με τα τρέχοντα στοιχεία του προϊόντος
          \item[3.1.3 ] Ο χρήστης τροποποιεί τα επιθυμητά πεδία
          \item[3.1.4 ] Ο χρήστης υποβάλλει τις αλλαγές
          \item[3.1.5 ] Το σύστημα επικυρώνει και αποθηκεύει τις αλλαγές
          \item[3.1.6 ] Το σύστημα επιστρέφει στη λίστα προϊόντων
        \end{enumerate}
  \item[2 ] Απενεργοποίηση προϊόντος:
        \begin{enumerate}
          \item [3.2.1 ] Ο χρήστης επιλέγει ένα προϊόν από τη λίστα
          \item [3.2.2 ] Ο χρήστης επιλέγει την επιλογή απενεργοποίησης
          \item [3.2.3 ] Το σύστημα ζητά επιβεβαίωση
          \item [3.2.4 ] Ο χρήστης επιβεβαιώνει την απενεργοποίηση
          \item [3.2.5 ] Το σύστημα μαρκάρει το προϊόν ως ανενεργό χωρίς να το διαγράψει
          \item [3.2.6 ] Το σύστημα ενημερώνει τη λίστα προϊόντων
        \end{enumerate}
  \item[3 ] Φιλτράρισμα προϊόντων:
        \begin{enumerate}
          \item [3.3.1 ] Ο χρήστης επιλέγει κριτήρια φιλτραρίσματος %(κατηγορία, προμηθευτής, κατάσταση)
          \item [3.3.2 ] Το σύστημα εφαρμόζει τα φίλτρα στη λίστα προϊόντων
          \item [3.3.3 ] Το σύστημα εμφανίζει τα φιλτραρισμένα αποτελέσματα
        \end{enumerate}
  \item[4 ] Αναζήτηση προϊόντων:
        \begin{enumerate}
          \item [3.4.1 ] Ο χρήστης εισάγει όρους αναζήτησης στο πεδίο αναζήτησης
          \item [3.4.2 ] Το σύστημα αναζητά προϊόντα που ταιριάζουν με τους όρους
          \item [3.4.3 ] Το σύστημα εμφανίζει τα αποτελέσματα της αναζήτησης
        \end{enumerate}
        % \item[5 ] Μαζική εισαγωγή προϊόντων:
        %       \begin{enumerate}
        %         \item [5.1 ] Ο χρήστης επιλέγει την επιλογή μαζικής εισαγωγής
        %         \item [5.2 ] Το σύστημα εμφανίζει οδηγίες για το απαιτούμενο format του αρχείου
        %         \item [5.3 ] Ο χρήστης μεταφορτώνει αρχείο CSV/Excel με τα προϊόντα
        %         \item [5.4 ] Το σύστημα επικυρώνει το αρχείο και εμφανίζει προεπισκόπηση των δεδομένων
        %         \item [5.5 ] Ο χρήστης επιβεβαιώνει την εισαγωγή
        %         \item [5.6 ] Το σύστημα εισάγει τα προϊόντα και εμφανίζει αναφορά με τα αποτελέσματα
        %       \end{enumerate}
        % \item[6 ] Διπλότυπος κωδικός προϊόντος:
        %       \begin{enumerate}
        %         \item [6.1 ] Ο χρήστης εισάγει κωδικό προϊόντος που ήδη υπάρχει
        %         \item [6.2 ] Το σύστημα ανιχνεύει τον διπλότυπο κωδικό
        %         \item [6.3 ] Το σύστημα εμφανίζει μήνυμα σφάλματος
        %         \item [6.4 ] Ο χρήστης διορθώνει τον κωδικό και συνεχίζει
        %       \end{enumerate}
        % \item[7 ] Εξαγωγή καταλόγου προϊόντων:
        %       \begin{enumerate}
        %         \item [7.1 ] Ο χρήστης επιλέγει την επιλογή εξαγωγής
        %         \item [7.2 ] Το σύστημα προσφέρει επιλογές μορφής εξαγωγής (CSV, Excel, PDF)
        %         \item [7.3 ] Ο χρήστης επιλέγει την επιθυμητή μορφή
        %         \item [7.4 ] Το σύστημα δημιουργεί και παρέχει το αρχείο για λήψη
        %       \end{enumerate}
\end{enumerate}

\subsection{Ειδικές απαιτήσεις}
\begin{itemize}
  \item Υποστήριξη προϊόντων με πολλαπλές παραλλαγές (π.χ. διαφορετικά μεγέθη, χρώματα)
  \item Δυνατότητα προσθήκης φωτογραφιών για κάθε προϊόν
  \item Υποστήριξη κωδικών barcode, QR και άλλων προτύπων αναγνώρισης
  \item Δυνατότητα ορισμού σχέσεων μεταξύ προϊόντων (συμβατότητα, εναλλακτικά προϊόντα)
  \item Καταγραφή και παρακολούθηση αλλαγών στα προϊόντα (versioning)
\end{itemize}

\subsubsection{Μη λειτουργικές απαιτήσεις}
\begin{itemize}
  \item Γρήγορη απόκριση στις αναζητήσεις (<1 δευτερόλεπτο) ακόμα και σε μεγάλο κατάλογο προϊόντων
  \item Υποστήριξη ταυτόχρονης επεξεργασίας από πολλαπλούς χρήστες χωρίς συγκρούσεις
  \item Διασφάλιση ακεραιότητας δεδομένων στις συναλλαγές της βάσης δεδομένων
  \item Ταχεία απόκριση στην προσθήκη και ενημέρωση προϊόντων (<2 δευτερόλεπτα)
\end{itemize}

\subsubsection{Περιβάλλον}
\begin{itemize}
  \item Διαδικτυακή διεπαφή με δυνατότητα προσαρμογής σε διαφορετικές συσκευές
  \item Υποστήριξη για φορητές συσκευές με σαρωτές barcode
  \item Ενσωμάτωση με το συνολικό σύστημα διαχείρισης αποθήκης
\end{itemize}

\subsection{Κατάσταση εισόδου}
\begin{itemize}
  \item Ο χρήστης έχει συνδεθεί στο σύστημα με τα απαραίτητα δικαιώματα
  \item Ο χρήστης έχει πρόσβαση στην ενότητα διαχείρισης προϊόντων
\end{itemize}

\subsection{Κάτασταση εξόδου}
\begin{itemize}
  \item Τα δεδομένα των προϊόντων έχουν ενημερωθεί επιτυχώς
  \item Ο χρήστης επιστρέφει στη λίστα προϊόντων ή στο κεντρικό μενού διαχείρισης αποθήκης
  \item Οι αλλαγές έχουν καταγραφεί στο ιστορικό του συστήματος
\end{itemize}

\subsection{Σημεία επέκτασης}
\begin{itemize}
  \item Ενσωμάτωση με σύστημα προμηθειών για αυτόματη ενημέρωση τιμών από προμηθευτές
  \item Προσθήκη δυνατότητας σάρωσης προϊόντων για αυτόματη καταχώρηση
  \item Υποστήριξη για πολυγλωσσικές περιγραφές προϊόντων
  \item Διασύνδεση με ηλεκτρονικό κατάστημα για συγχρονισμό προϊόντων
\end{itemize}

\section{Perform Inventory}

\subsection{Σύντομη Περιγραφή}
Η περίπτωση χρήσης "Perform Inventory" επιτρέπει στο εξουσιοδοτημένο προσωπικό να διεξάγει απογραφές του αποθέματος της κτηνιατρικής κλινικής. Αυτό περιλαμβάνει την καταμέτρηση των φυσικών ποσοτήτων των προϊόντων, τη σύγκριση με τις καταγεγραμμένες ποσότητες στο σύστημα, την καταγραφή και διόρθωση αποκλίσεων, και την παραγωγή αναφορών απογραφής για λογιστικούς και διαχειριστικούς σκοπούς. % chktex 18

\subsection{Χειριστές}
\begin{itemize}
  \item Διαχειριστής συστήματος
  \item Κτηνίατρος
  \item Γραμματέας
\end{itemize}

\subsection{Γεγονός Έναρξης}
Η περίπτωση χρήσης ξεκινά όταν:
\begin{itemize}
  \item Ένας εξουσιοδοτημένος χρήστης επιλέγει την επιλογή "Perform Inventory" από την ενότητα "Warehouse Management" % chktex 18
  \item Έχει προγραμματιστεί τακτική απογραφή (μηνιαία, τριμηνιαία, ετήσια)
\end{itemize}

\subsection{Ροή γεγονότων}

\subsubsection{Βασική Ροή}
\begin{enumerate}
  \item Ο χρήστης επιλέγει την επιλογή "Perform Inventory" από το μενού διαχείρισης αποθήκης. % chktex 18
  \item Το σύστημα κάνει πλήρη απογραφή.
  \item Το σύστημα δημιουργεί μια νέα απογραφή και εμφανίζει λίστα με όλα τα προϊόντα και τις καταγεγραμμένες ποσότητες.
  \item Ο χρήστης εισάγει τις πραγματικές ποσότητες που καταμετρά για κάθε προϊόν.
  \item Το σύστημα υπολογίζει αυτόματα τις αποκλίσεις μεταξύ των καταγεγραμμένων και των πραγματικών ποσοτήτων.
  \item Ο χρήστης επιβεβαιώνει την ολοκλήρωση της απογραφής.
  \item Το σύστημα ενημερώνει τα αποθέματα με τις νέες ποσότητες και καταγράφει τις αποκλίσεις.
  \item Το σύστημα παράγει αναφορά απογραφής με συγκεντρωτικά στοιχεία και λεπτομέρειες αποκλίσεων.
\end{enumerate}

\subsubsection{Εναλλακτικές Ροές}
\begin{enumerate}
  \item[1 ] Μερική απογραφή:
        \begin{enumerate}
          \item[2.1.1 ] Ο χρήστης επιλέγει μερική απογραφή
          \item[2.1.2 ] Το σύστημα ζητά από τον χρήστη να καθορίσει κριτήρια επιλογής προϊόντων
          \item[2.1.3 ] Ο χρήστης επιλέγει κριτήρια %(κατηγορία, προμηθευτής, θέση αποθήκης)
          \item[2.1.4 ] Το σύστημα εμφανίζει μόνο τα προϊόντα που πληρούν τα κριτήρια
          \item[2.1.5 ] Η διαδικασία συνεχίζεται από το βήμα 4 της βασικής ροής
        \end{enumerate}
        % \item[2 ] Απογραφή με σαρωτή barcode:
        %       \begin{enumerate}
        %         \item [2.1 ] Μετά το βήμα 4, ο χρήστης επιλέγει τη λειτουργία σάρωσης
        %         \item [2.2 ] Ο χρήστης σαρώνει το barcode προϊόντος
        %         \item [2.3 ] Το σύστημα αναγνωρίζει το προϊόν και ζητά την καταμετρημένη ποσότητα
        %         \item [2.4 ] Ο χρήστης εισάγει την ποσότητα και συνεχίζει με το επόμενο προϊόν
        %         \item [2.5 ] Η διαδικασία επαναλαμβάνεται μέχρι να ολοκληρωθεί η απογραφή
        %       \end{enumerate}
  \item[2 ] Αναβολή απογραφής:
        \begin{enumerate}
          \item [2.1 ] Ο χρήστης επιλέγει την προσωρινή αποθήκευση της απογραφής σε οποιοδήποτε σημείο
          \item [2.2 ] Το σύστημα αποθηκεύει την τρέχουσα πρόοδο χωρίς να ενημερώσει τα αποθέματα
          \item [2.3 ] Ο χρήστης μπορεί αργότερα να επαναφέρει και να συνεχίσει την απογραφή
        \end{enumerate}
  \item[3 ] Διόρθωση μεγάλων αποκλίσεων:
        \begin{enumerate}
          \item [5.3.1 ] Tο σύστημα επισημαίνει προϊόντα με μεγάλες αποκλίσεις
          \item [5.3.2 ] Ο χρήστης επιλέγει επανέλεγχο των επισημασμένων προϊόντων
          \item [5.3.3 ] Ο χρήστης επιβεβαιώνει ή διορθώνει τις ποσότητες
          \item [5.3.4 ] Για κάθε σημαντική απόκλιση, ο χρήστης εισάγει αιτιολογία
          \item [5.3.5 ] Η διαδικασία συνεχίζεται από το βήμα 6 της βασικής ροής
        \end{enumerate}
        % \item[5 ] Εξαγωγή φύλλου απογραφής:
        %       \begin{enumerate}
        %         \item [5.1 ] Μετά το βήμα 4, ο χρήστης επιλέγει εξαγωγή σε αρχείο
        %         \item [5.2 ] Το σύστημα δημιουργεί αρχείο (Excel/CSV) με τα προϊόντα για καταμέτρηση
        %         \item [5.3 ] Ο χρήστης διεξάγει την απογραφή εκτός συστήματος
        %         \item [5.4 ] Ο χρήστης επιστρέφει και μεταφορτώνει το συμπληρωμένο αρχείο
        %         \item [5.5 ] Το σύστημα εισάγει τις καταμετρημένες ποσότητες
        %         \item [5.6 ] Η διαδικασία συνεχίζεται από το βήμα 6 της βασικής ροής
        %       \end{enumerate}
  \item[4 ] Ακύρωση απογραφής:
        \begin{enumerate}
          \item [4.1 ] Ο χρήστης επιλέγει ακύρωση της απογραφής σε οποιοδήποτε σημείο
          \item [4.2 ] Το σύστημα ζητά επιβεβαίωση για την ακύρωση
          \item [4.3 ] Ο χρήστης επιβεβαιώνει
          \item [4.4 ] Το σύστημα απορρίπτει όλες τις αλλαγές και επιστρέφει στο κεντρικό μενού
        \end{enumerate}
\end{enumerate}

\subsection{Ειδικές απαιτήσεις}
\begin{itemize}
  \item Δυνατότητα διεξαγωγής τμηματικής απογραφής για μεγάλες αποθήκες
  \item Υποστήριξη για φορητές συσκευές και σαρωτές barcode
  \item Αυτόματη αναγνώριση και επισήμανση ασυνήθιστων αποκλίσεων
  \item Διατήρηση πλήρους ιστορικού απογραφών για ελεγκτικούς σκοπούς
  \item Ταυτόχρονη διεξαγωγή απογραφής από πολλαπλούς χρήστες σε διαφορετικά τμήματα
\end{itemize}

\subsubsection{Μη λειτουργικές απαιτήσεις}
\begin{itemize}
  \item Δυνατότητα λειτουργίας σε συνθήκες περιορισμένης συνδεσιμότητας
  \item Ταχεία καταχώρηση μεγάλου όγκου δεδομένων χωρίς καθυστερήσεις
  \item Υψηλή ακρίβεια στους υπολογισμούς αποκλίσεων και αξιών αποθέματος
  \item Οι αναφορές πρέπει να παράγονται σε λιγότερο από 30 δευτερόλεπτα ακόμα και για μεγάλο αριθμό προϊόντων
\end{itemize}

\subsubsection{Περιβάλλον}
\begin{itemize}
  \item Διαδικτυακή διεπαφή προσαρμοσμένη για χρήση και σε tablet
  \item Εφαρμογή για φορητές συσκευές με υποστήριξη λειτουργίας offline
  \item Υποστήριξη για φορητούς σαρωτές και εκτυπωτές ετικετών
\end{itemize}

\subsection{Κατάσταση εισόδου}
\begin{itemize}
  \item Ο χρήστης έχει συνδεθεί στο σύστημα με τα απαραίτητα δικαιώματα
  \item Το σύστημα έχει τρέχουσα κατάσταση αποθέματος διαθέσιμη
  \item Δεν υπάρχει άλλη ενεργή απογραφή σε εξέλιξη (για πλήρη απογραφή)
\end{itemize}

\subsection{Κάτασταση εξόδου}
\begin{itemize}
  \item Το απόθεμα έχει ενημερωθεί με τις πραγματικές ποσότητες
  \item Έχει δημιουργηθεί αναφορά απογραφής με τις αποκλίσεις και τις αιτιολογίες
  \item Το ιστορικό απογραφών έχει ενημερωθεί
  \item Έχουν δημιουργηθεί οι σχετικές λογιστικές εγγραφές για τις διαφορές αποθέματος
\end{itemize}

\subsection{Σημεία επέκτασης}
\begin{itemize}
  \item Ενσωμάτωση τεχνολογίας RFID για αυτοματοποιημένη απογραφή
  \item Προσθήκη λειτουργίας οπτικής αναγνώρισης προϊόντων μέσω κάμερας
  \item Διασύνδεση με λογιστικό σύστημα για αυτόματη ενημέρωση αξίας αποθέματος
  \item Προσθήκη αλγορίθμων πρόβλεψης για βελτιστοποίηση της συχνότητας απογραφών
\end{itemize}

\section{Manage Orders}

\subsection{Σύντομη Περιγραφή}
Η περίπτωση χρήσης "Manage Orders" επιτρέπει στο εξουσιοδοτημένο προσωπικό να διαχειρίζεται τις παραγγελίες προϊόντων προς τους προμηθευτές. Περιλαμβάνει τη δημιουργία νέων παραγγελιών, την παρακολούθηση της κατάστασής τους, την παραλαβή προϊόντων και την ενημέρωση του αποθέματος. Το σύστημα υποστηρίζει αυτόματες προτάσεις παραγγελίας βάσει ελάχιστων ορίων αποθέματος και ιστορικού κατανάλωσης. % chktex 18

\subsection{Χειριστές}
\begin{itemize}
  \item Διαχειριστής συστήματος
  \item Κτηνίατρος
  \item Γραμματέας
\end{itemize}

\subsection{Γεγονός Έναρξης}
Η περίπτωση χρήσης ξεκινά όταν:
\begin{itemize}
  \item Ένας εξουσιοδοτημένος χρήστης επιλέγει την επιλογή "Manage Orders" από την ενότητα "Warehouse Management" % chktex 18
  \item Το σύστημα εντοπίζει προϊόντα που έχουν φτάσει στο ελάχιστο όριο αποθέματος
\end{itemize}

\subsection{Ροή γεγονότων}

\subsubsection{Βασική Ροή}
\begin{enumerate}
  \item Ο χρήστης επιλέγει την ενότητα "Manage Orders". % chktex 18
  \item Το σύστημα εμφανίζει λίστα με όλες τις τρέχουσες παραγγελίες και την κατάστασή τους.
  \item Ο χρήστης επιλέγει τη δημιουργία νέας παραγγελίας.
  \item Το σύστημα εμφανίζει τη φόρμα παραγγελίας με προτεινόμενα προϊόντα βάσει επιπέδων αποθέματος.
  \item Ο χρήστης επιλέγει προϊόντα και καθορίζει ποσότητες.
  \item Ο χρήστης επιλέγει προμηθευτή για τα προϊόντα.
  \item Το σύστημα υπολογίζει το συνολικό κόστος και εμφανίζει προεπισκόπηση της παραγγελίας.
  \item Ο χρήστης επιβεβαιώνει την παραγγελία.
  \item Το σύστημα καταχωρεί την παραγγελία και ενημερώνει τη λίστα παραγγελιών.
\end{enumerate}

\subsubsection{Εναλλακτικές Ροές}
\begin{enumerate}
  \item[1 ] Παραλαβή παραγγελίας:
        \begin{enumerate}
          \item[3.1.1 ] Ο χρήστης επιλέγει μια παραγγελία σε κατάσταση "Pending" % chktex 18
          \item[3.1.2 ] Ο χρήστης επιλέγει την επιλογή "Receive Order" % chktex 18
          \item[3.1.3 ] Το σύστημα εμφανίζει φόρμα παραλαβής με τα αναμενόμενα προϊόντα
          \item[3.1.4 ] Ο χρήστης καταχωρεί τις παραληφθείσες ποσότητες και ημερομηνίες λήξης
          \item[3.1.5 ] Το σύστημα ενημερώνει το απόθεμα και την κατάσταση της παραγγελίας
          \item[3.1.6 ] Το σύστημα εμφανίζει λίστα με όλες τις τρέχουσες παραγγελίες και την κατάστασή τους.
        \end{enumerate}
  \item[2 ] Μερική παραλαβή:
        \begin{enumerate}
          \item[3.1.2.1 ] Κατά την παραλαβή, ο χρήστης σημειώνει ότι κάποια προϊόντα δεν παραλήφθηκαν
          \item[3.1.2.2 ] Το σύστημα δημιουργεί νέα παραγγελία για τα υπολειπόμενα προϊόντα
          \item[3.1.2.3 ] Ο χρήστης επιβεβαιώνει την παραλαβή των διαθέσιμων προϊόντων
          \item[3.1.2.4 ] Το σύστημα ενημερώνει το απόθεμα και διατηρεί την εκκρεμότητα για τα υπόλοιπα
          \item[3.1.2.5] Το σύστημα εμφανίζει λίστα με όλες τις τρέχουσες παραγγελίες και την κατάστασή τους.
        \end{enumerate}
  \item[3 ] Ακύρωση παραγγελίας:
        \begin{enumerate}
          \item[3.3.1 ] Ο χρήστης επιλέγει μια παραγγελία σε εκκρεμότητα
          \item[3.3.2 ] Ο χρήστης επιλέγει την ακύρωση της παραγγελίας
          \item[3.3.3 ] Το σύστημα ζητά επιβεβαίωση και αιτιολογία
          \item[3.3.4 ] Ο χρήστης επιβεβαιώνει την ακύρωση
          \item[3.3.5 ] Το σύστημα ενημερώνει την κατάσταση της παραγγελίας σε "Cancelled" % chktex 18
          \item[3.3.6] Το σύστημα εμφανίζει λίστα με όλες τις τρέχουσες παραγγελίες και την κατάστασή τους.
        \end{enumerate}
  \item[4 ] Τροποποίηση παραγγελίας:
        \begin{enumerate}
          \item[3.4.1 ] Ο χρήστης επιλέγει μια παραγγελία που δεν έχει αποσταλεί
          \item[3.4.2 ] Ο χρήστης τροποποιεί ποσότητες ή προσθέτει/αφαιρεί προϊόντα
          \item[3.4.3 ] Το σύστημα επανυπολογίζει το συνολικό κόστος
          \item[3.4.4 ] Ο χρήστης αποθηκεύει τις αλλαγές
          \item[3.4.5 ] Το σύστημα εμφανίζει λίστα με όλες τις τρέχουσες παραγγελίες και την κατάστασή τους.
        \end{enumerate}
  \item[5 ] Αυτόματη δημιουργία παραγγελίας:
        \begin{enumerate}
          \item[3.5.1 ] Το σύστημα εντοπίζει προϊόντα κάτω από το ελάχιστο όριο
          \item[3.5.2 ] Το σύστημα δημιουργεί προτεινόμενη παραγγελία και ειδοποιεί τον χρήστη
          \item[3.5.3 ] Ο χρήστης ελέγχει και τροποποιεί την προτεινόμενη παραγγελία
          \item[3.5.4 ] Ο χρήστης επιβεβαιώνει την παραγγελία
          \item[3.5.5 ] Το σύστημα εμφανίζει λίστα με όλες τις τρέχουσες παραγγελίες και την κατάστασή τους.
        \end{enumerate}
  \item[6 ] Ακύρωση διαδικασίας:
        \begin{enumerate}
          \item[6.1 ] Ο χρήστης επιλέγει την ακύρωση της διαδικασίας σε οποιοδήποτε σημείο
          \item[6.2 ] Το σύστημα ζητά επιβεβαίωση για την ακύρωση
          \item[6.3 ] Ο χρήστης επιβεβαιώνει
          \item[6.4 ] Το σύστημα απορρίπτει όλες τις αλλαγές και εμφανίζει λίστα με όλες τις τρέχουσες παραγγελίες και την κατάστασή τους.
        \end{enumerate}
  \item[7 ] Αναζήτηση παραγγελίας:
        \begin{enumerate}
          \item[3.7.1 ] Ο χρήστης επιλέγει την αναζήτηση παραγγελίας από τη λίστα
          \item[3.7.2 ] Το σύστημα εμφανίζει φόρμα αναζήτησης με φίλτρα %(ημερομηνία, προμηθευτής, κατάσταση)
          \item[3.7.3 ] Ο χρήστης εισάγει τα κριτήρια αναζήτησης
          \item[3.7.4 ] Το σύστημα εμφανίζει τα αποτελέσματα της αναζήτησης στην οθόνη
        \end{enumerate}
  \item[8 ] Φιλτράρισμα παραγγελιών:
        \begin{enumerate}
          \item[3.8.1 ] Ο χρήστης επιλέγει φίλτρα για την προβολή παραγγελιών %(προμηθευτής, κατάσταση, ημερομηνία)
          \item[3.8.2 ] Το σύστημα εμφανίζει τις παραγγελίες που πληρούν τα κριτήρια
        \end{enumerate}
\end{enumerate}

\subsection{Ειδικές απαιτήσεις}
\begin{itemize}
  \item Δυνατότητα παρακολούθησης πολλαπλών παραγγελιών ταυτόχρονα
  \item Αυτόματος υπολογισμός προτεινόμενων ποσοτήτων βάσει ιστορικού
  \item Υποστήριξη διαφορετικών τύπων παραγγελιών (τακτικές, έκτακτες, επείγουσες)
  \item Διαχείριση πολλαπλών προμηθευτών και τιμοκαταλόγων
  \item Αυτόματη ενημέρωση αποθέματος κατά την παραλαβή
\end{itemize}

\subsubsection{Μη λειτουργικές απαιτήσεις}
\begin{itemize}
  \item Ταχεία επεξεργασία παραγγελιών (<2 δευτερόλεπτα για αποθήκευση)
  \item Υποστήριξη ταυτόχρονης επεξεργασίας από πολλαπλούς χρήστες
  \item Διατήρηση ιστορικού παραγγελιών για τουλάχιστον 5 έτη
  \item Αυτόματη δημιουργία αντιγράφων ασφαλείας μετά από κάθε σημαντική ενέργεια
\end{itemize}

\subsubsection{Περιβάλλον}
\begin{itemize}
  \item Διαδικτυακή διεπαφή συμβατή με όλους τους σύγχρονους browsers
  \item Υποστήριξη για φορητές συσκευές και tablets
  \item Δυνατότητα εκτύπωσης παραγγελιών και αναφορών
\end{itemize}

\subsection{Κατάσταση εισόδου}
\begin{itemize}
  \item Ο χρήστης έχει συνδεθεί με τα κατάλληλα δικαιώματα
  \item Υπάρχει ενημερωμένη λίστα προμηθευτών και προϊόντων
  \item Το σύστημα έχει τρέχουσες πληροφορίες αποθέματος
\end{itemize}

\subsection{Κάτασταση εξόδου}
\begin{itemize}
  \item Οι παραγγελίες έχουν καταχωρηθεί ή ενημερωθεί
  \item Το απόθεμα έχει ενημερωθεί (σε περίπτωση παραλαβών)
  \item Έχουν δημιουργηθεί οι απαραίτητες ειδοποιήσεις και αναφορές
\end{itemize}

\subsection{Σημεία επέκτασης}
\begin{itemize}
  \item Ενσωμάτωση με συστήματα προμηθευτών για αυτόματη αποστολή παραγγελιών
  \item Προσθήκη συστήματος παρακολούθησης παράδοσης (tracking)
  \item Υποστήριξη για αυτόματη τιμολόγηση και πληρωμές
  \item Ενσωμάτωση με συστήματα πρόβλεψης ζήτησης
\end{itemize}

\section{Generate Reports}

\subsection{Σύντομη Περιγραφή}
Η περίπτωση χρήσης "Generate Reports" επιτρέπει στους χρήστες να δημιουργούν και να προβάλλουν διάφορους τύπους αναφορών σχετικά με τη διαχείριση αποθήκης. Οι αναφορές μπορούν να περιλαμβάνουν στατιστικά αποθέματος, ιστορικό κινήσεων, αξία αποθέματος, προβλέψεις ζήτησης και άλλα σημαντικά δεδομένα για τη λήψη αποφάσεων. % chktex 18

\subsection{Χειριστές}
\begin{itemize}
  \item Διαχειριστής συστήματος
  \item Κτηνίατρος
  \item Γραμματέας
\end{itemize}

\subsection{Γεγονός Έναρξης}
Η περίπτωση χρήσης ξεκινά όταν:
\begin{itemize}
  \item Ένας χρήστης επιλέγει την επιλογή "Generate Reports" % chktex 18
  \item Απαιτείται η δημιουργία περιοδικής αναφοράς
\end{itemize}

\subsection{Ροή γεγονότων}

\subsubsection{Βασική Ροή}
\begin{enumerate}
  \item Ο χρήστης επιλέγει την ενότητα "Generate Reports". % chktex 18
  \item Το σύστημα εμφανίζει λίστα με διαθέσιμους τύπους αναφορών.
  \item Ο χρήστης επιλέγει τον επιθυμητό τύπο αναφοράς.
  \item Το σύστημα ζητά παραμέτρους για την αναφορά% (χρονικό διάστημα, κατηγορίες προϊόντων, κτλ.).
  \item Ο χρήστης συμπληρώνει τις παραμέτρους.
  \item Το σύστημα δημιουργεί την αναφορά.
  \item Το σύστημα εμφανίζει προεπισκόπηση της αναφοράς.
  \item Ο χρήστης επιλέγει τη μορφή εξαγωγής %(PDF, Excel, κτλ.).
  \item Το σύστημα παράγει και παρέχει την αναφορά στην επιλεγμένη μορφή.
  \item Το σύστημα ενημερώνει τον χρήστη για την επιτυχή ολοκλήρωση της διαδικασίας και επιστρέφει στη λίστα προϊόντων.
\end{enumerate}

\subsubsection{Εναλλακτικές Ροές}
\begin{enumerate}
  \item[1 ] Προγραμματισμένη αναφορά:
        \begin{enumerate}
          \item[3.1.1 ] Ο χρήστης επιλέγει δημιουργία προγραμματισμένης αναφοράς
          \item[3.1.2 ] Το σύστημα εμφανίζει επιλογές προγραμματισμού
          \item[3.1.3 ] Ο χρήστης καθορίζει συχνότητα και παραμέτρους
          \item[3.1.4 ] Το σύστημα αποθηκεύει τον προγραμματισμό
          \item[3.1.5 ] Το σύστημα θα δημιουργεί αυτόματα την αναφορά στα καθορισμένα διαστήματα
        \end{enumerate}
  % \item[2 ] Προσαρμοσμένη αναφορά:
  %       \begin{enumerate}
  %         \item[2.1 ] Ο χρήστης επιλέγει δημιουργία προσαρμοσμένης αναφοράς
  %         \item[2.2 ] Το σύστημα εμφανίζει διαθέσιμα πεδία και μετρικές
  %         \item[2.3 ] Ο χρήστης επιλέγει τα επιθυμητά στοιχεία
  %         \item[2.4 ] Το σύστημα δημιουργεί την προσαρμοσμένη αναφορά
  %       \end{enumerate}
  % \item[3 ] Αποθήκευση προτύπου:
  %       \begin{enumerate}
  %         \item[3.1 ] Μετά τη δημιουργία αναφοράς, ο χρήστης επιλέγει αποθήκευση ως πρότυπο
  %         \item[3.2 ] Το σύστημα ζητά όνομα για το πρότυπο
  %         \item[3.3 ] Ο χρήστης εισάγει όνομα και περιγραφή
  %         \item[3.4 ] Το σύστημα αποθηκεύει το πρότυπο για μελλοντική χρήση
  %       \end{enumerate}
  % \item[4 ] Κοινή χρήση αναφοράς:
  %       \begin{enumerate}
  %         \item[4.1 ] Ο χρήστης επιλέγει κοινή χρήση της αναφοράς
  %         \item[4.2 ] Το σύστημα εμφανίζει επιλογές κοινής χρήσης
  %         \item[4.3 ] Ο χρήστης επιλέγει παραλήπτες και δικαιώματα
  %         \item[4.4 ] Το σύστημα αποστέλλει την αναφορά ή παρέχει πρόσβαση
  %      \end{enumerate}
  \item[2 ] Σφάλμα κατά τη δημιουργία αναφοράς:
        \begin{enumerate}
          \item[3.2.1 ] Το σύστημα δεν μπορεί να δημιουργήσει την αναφορά λόγω εσωτερικού σφάλματος
          \item[3.2.2 ] Το σύστημα εμφανίζει μήνυμα σφάλματος και προτείνει λύσεις
          \item[3.2.3 ] Το σύστημα επιστρέφει τον χρήστη βήμα 2 της βασικής ροής
        \end{enumerate}
  \item[3 ] Ακύρωση διαδικασίας:
        \begin{enumerate}
          \item[3.3.1 ] Ο χρήστης επιλέγει την ακύρωση της διαδικασίας σε οποιοδήποτε σημείο
          \item[3.3.2 ] Το σύστημα ζητά επιβεβαίωση για την ακύρωση
          \item[3.3.3 ] Ο χρήστης επιβεβαιώνει
          \item[3.3.4 ] Το σύστημα απορρίπτει όλες τις αλλαγές και επιστρέφει στη λίστα αναφορών
        \end{enumerate}
\end{enumerate}

\subsection{Ειδικές απαιτήσεις}
\begin{itemize}
  \item Υποστήριξη πολλαπλών μορφών εξαγωγής (PDF, Excel, CSV, HTML)
  \item Δυνατότητα προσαρμογής της εμφάνισης των αναφορών
  \item Υποστήριξη γραφημάτων και οπτικοποιήσεων δεδομένων
  \item Αυτόματος υπολογισμός βασικών μετρικών και KPIs
  \item Δυνατότητα αποθήκευσης και επαναχρησιμοποίησης προτύπων
\end{itemize}

\subsubsection{Μη λειτουργικές απαιτήσεις}
\begin{itemize}
  \item Γρήγορη δημιουργία αναφορών (<30 δευτερόλεπτα για τυπικές αναφορές)
  \item Υποστήριξη μεγάλου όγκου δεδομένων χωρίς επιπτώσεις στην απόδοση
  \item Ακριβής υπολογισμός όλων των μετρικών και στατιστικών
  \item Ασφαλής αποθήκευση και πρόσβαση στις αναφορές
\end{itemize}

\subsubsection{Περιβάλλον}
\begin{itemize}
  \item Διαδικτυακή διεπαφή για δημιουργία και προβολή αναφορών
  \item Υποστήριξη εκτύπωσης σε διάφορες συσκευές
  \item Συμβατότητα με δημοφιλή προγράμματα επεξεργασίας αρχείων
\end{itemize}

\subsection{Κατάσταση εισόδου}
\begin{itemize}
  \item Ο χρήστης έχει τα απαραίτητα δικαιώματα
  \item Τα δεδομένα για την αναφορά είναι διαθέσιμα
  \item Το σύστημα έχει επαρκείς πόρους για την επεξεργασία
\end{itemize}

\subsection{Κάτασταση εξόδου}
\begin{itemize}
  \item Η αναφορά έχει δημιουργηθεί επιτυχώς
  \item Η αναφορά έχει αποθηκευτεί ή εξαχθεί στην επιθυμητή μορφή
  \item Έχει καταγραφεί η δημιουργία της αναφοράς στο ιστορικό
\end{itemize}

\subsection{Σημεία επέκτασης}
\begin{itemize}
  \item Προσθήκη προηγμένων αναλυτικών στοιχείων και προβλέψεων
  \item Ενσωμάτωση με συστήματα επιχειρηματικής ευφυΐας
  \item Υποστήριξη για δυναμικά dashboards
  \item Αυτοματοποιημένη διανομή αναφορών
\end{itemize}

\section{Monitor Expiration}

\subsection{Σύντομη Περιγραφή}
Η περίπτωση χρήσης "Monitor Expiration" επιτρέπει την παρακολούθηση και διαχείριση των ημερομηνιών λήξης των προϊόντων στην αποθήκη. Το σύστημα παρακολουθεί τις ημερομηνίες λήξης, ειδοποιεί για προϊόντα που πλησιάζουν στη λήξη τους και βοηθά στη διαχείριση της απόσυρσης ληγμένων προϊόντων. % chktex 18

\subsection{Χειριστές}
\begin{itemize}
  \item Διαχειριστής συστήματος
  \item Κτηνίατρος
  \item Γραμματέας
\end{itemize}

\subsection{Γεγονός Έναρξης}
Η περίπτωση χρήσης ξεκινά όταν:
\begin{itemize}
  \item Ένας χρήστης επιλέγει την επιλογή "Monitor Expiration" % chktex 18
  \item Το σύστημα εντοπίζει προϊόντα που πλησιάζουν στη λήξη τους
\end{itemize}

\subsection{Ροή γεγονότων}

\subsubsection{Βασική Ροή}
\begin{enumerate}
  \item Ο χρήστης επιλέγει την ενότητα "Monitor Expiration". % chktex 18
  \item Το σύστημα εμφανίζει λίστα προϊόντων ταξινομημένη κατά ημερομηνία λήξης.
  \item Το σύστημα επισημαίνει με χρωματικούς κώδικες τα προϊόντα βάσει εγγύτητας λήξης.
  \item Ο χρήστης επιλέγει προϊόντα προς απόσυρση.
  \item Το σύστημα ζητά επιβεβαίωση για την απόσυρση.
  \item Ο χρήστης επιβεβαιώνει την απόσυρση.
  \item Το σύστημα ενημερώνει το απόθεμα και καταγράφει την απόσυρση.
  \item Το σύστημα δημιουργεί αναφορά αποσυρθέντων προϊόντων.
\end{enumerate}

\subsubsection{Εναλλακτικές Ροές}
\begin{enumerate}
  \item[1 ] Ρύθμιση ειδοποιήσεων:
        \begin{enumerate}
          \item[4.1.1 ] Ο χρήστης επιλέγει ρύθμιση ειδοποιήσεων λήξης
          \item[4.1.2 ] Το σύστημα εμφανίζει τις τρέχουσες ρυθμίσεις
          \item[4.1.3 ] Ο χρήστης καθορίζει τα χρονικά όρια ειδοποίησης
          \item[4.1.4 ] Ο χρήστης επιλέγει τους τύπους ειδοποιήσεων
          \item[4.1.5 ] Το σύστημα αποθηκεύει τις νέες ρυθμίσεις και γυρνάει στη λίστα προϊόντων
        \end{enumerate}
  % \item[3 ] Καταγραφή αιτιολογίας:
  %       \begin{enumerate}
  %         \item[3.1 ] Κατά την απόσυρση, ο χρήστης επιλέγει καταγραφή αιτιολογίας
  %         \item[3.2 ] Το σύστημα εμφανίζει φόρμα καταγραφής
  %         \item[3.3 ] Ο χρήστης εισάγει την αιτιολογία απόσυρσης
  %         \item[3.4 ] Το σύστημα αποθηκεύει την αιτιολογία μαζί με την απόσυρση
  %       \end{enumerate}
  \item[2 ] Προβολή ιστορικού:
        \begin{enumerate}
          \item[4.2.1 ] Ο χρήστης επιλέγει προβολή ιστορικού αποσύρσεων
          \item[4.2.2 ] Το σύστημα εμφανίζει το ιστορικό με φίλτρα
          \item[4.2.3 ] Ο χρήστης τελειώνει και γυρνάει στη λίστα προϊόντων
        \end{enumerate}
  \item[3 ] Ακύρωση διαδικασίας:
        \begin{enumerate}
          \item[6.3.1 ] Ο χρήστης επιλέγει την ακύρωση της διαδικασίας σε οποιοδήποτε σημείο
          \item[6.3.2 ] Το σύστημα ζητά επιβεβαίωση για την ακύρωση
          \item[6.3.3 ] Ο χρήστης επιβεβαιώνει
          \item[6.3.4 ] Το σύστημα απορρίπτει όλες τις αλλαγές και επιστρέφει στη λίστα προϊόντων
        \end{enumerate}
\end{enumerate}

\subsection{Ειδικές απαιτήσεις}
\begin{itemize}
  \item Αυτόματη παρακολούθηση ημερομηνιών λήξης
  \item Υποστήριξη διαφορετικών επιπέδων προειδοποίησης
  \item Δυνατότητα ορισμού διαφορετικών κανόνων ανά κατηγορία προϊόντων
  \item Αυτόματη δημιουργία αναφορών απόσυρσης
  \item Ιχνηλασιμότητα αποσυρθέντων προϊόντων
\end{itemize}

\subsubsection{Μη λειτουργικές απαιτήσεις}
\begin{itemize}
  \item Έγκαιρη αποστολή ειδοποιήσεων χωρίς καθυστερήσεις
  \item Ακριβής υπολογισμός ημερομηνιών και περιόδων
  \item Αξιόπιστη λειτουργία του συστήματος παρακολούθησης
  \item Αποδοτική διαχείριση μεγάλου όγκου προϊόντων
\end{itemize}

\subsubsection{Περιβάλλον}
\begin{itemize}
  \item Διαδικτυακή διεπαφή με εύκολη πλοήγηση
  \item Υποστήριξη ειδοποιήσεων μέσω email και SMS
  \item Συμβατότητα με φορητές συσκευές
\end{itemize}

\subsection{Κατάσταση εισόδου}
\begin{itemize}
  \item Ο χρήστης έχει τα απαραίτητα δικαιώματα
  \item Υπάρχουν καταχωρημένες ημερομηνίες λήξης για τα προϊόντα
  \item Το σύστημα παρακολούθησης είναι ενεργό
\end{itemize}

\subsection{Κάτασταση εξόδου}
\begin{itemize}
  \item Ενημερωμένη κατάσταση προϊόντων μετά τις αποσύρσεις
  \item Καταγεγραμμένες ενέργειες στο ιστορικό
  \item Δημιουργημένες αναφορές απόσυρσης
\end{itemize}

\subsection{Σημεία επέκτασης}
\begin{itemize}
  \item Προσθήκη προβλέψεων για βελτιστοποίηση διαχείρισης αποθέματος
  \item Αυτοματοποιημένη δημιουργία παραγγελιών αντικατάστασης
  \item Ενσωμάτωση με συστήματα διαχείρισης αποβλήτων
  \item Προσθήκη αναλύσεων κόστους λόγω λήξεων
\end{itemize}

\end{document} % chktex 17